\documentclass{article}
\usepackage{../refalg}

\begin{document}
\Makepagesectionhead{MATH 594}{Group}{ARessegetes Stery}

\tableofcontents  
\newpage

\section{Group Preliminaries}

\begin{definition}[Group]
    A \textbf{group} is a set $G$ together with a binary operation $G \times G \to G$, often written $(a, b) \mapsto a \cdot b$ or simply $ab$, s.t. the following properties are satisfied:
    \begin{enumerate}
        \item \emph{Associativity:} $(ab)c = a(bc)$ for all $a, b, c \in G$.
        \item \emph{Existence of Identity}: There exists $e = e_G \in G$ s.t. $\forall a \in G, ae = a = ea$.
        \item \emph{Existence of Inverse}: For all $a \in G$. there exists $b \in G$ s.t. $ab = e = ba$.
    \end{enumerate}
    Furthermore, if the operation is commutative, i.e. for all $a, b \in G$, $ab = ba$, then the group is \textbf{commutative}, or \textbf{abelian}.
\end{definition}

\begin{notation}
    If the group $G$ is abelian, then the operation is often represented in additive notations (with operation denoted as ``$+$'', and inverse of $a \in G$ being $-a$).
\end{notation}

\begin{remark}
    One implicitly presented condition is that the operation of groups need to be closed within the set predefined. This is indicated by the signature of the operation, which should land in $G$. This often needs to be checked when the group structure is defined in some larger structure.
\end{remark}

\begin{remark}\label{rmk: group immediate results}
    From the definition of group there are some immediate facts/properties:
    \begin{enumerate}[label=\arabic*)]
        \item The identity in the group is unique. Suppose that there exist two identity elements $e$ and $e'$, then by rule 2 $e = ee' = e'$.
        \item For a given element in the group, the inverse of it is unique. Let $b$ and $b'$ both be the inverse of some $a \in G$. Then
        \[
            b = b(ab') = (ba)b' = b'    
        \]
        the uniqueness allows us to unambiguously denote the inverse of $a$ as $a^{-1}$. This also implies $(a^{-1})^{-1} = a$, as clearly by the previous process $a$ is the inverse of $a^{-1}$; and the inverse is unique.
        \item $(ab)^{-1} = b^{-1} a^{-1}$. By the uniqueness of the inverse element, it suffices to check that the claimed inverse satisfies rule 2. This is indeed the case as
        \[
            (ab)(b^{-1}a^{-1}) = (a(bb^{-1}))a^{-1} = (ae)a^{-1} = aa^{-1} = e
        \]
        and for multiplication in the other sequence the checking is similar. 
        \item For $a, b, c \in G$, then $ab = ac \implies b = c$; and $ba = ca \implies b = c$. This results directly from the fact that $a$ is invertible; and multiplying on the left/right, respectively, $a$, gives the desired result.
    \end{enumerate}
\end{remark}

\begin{remark}\label{rmk: Z to group element}
    The associativity of operation in the groups gives the unambiguity of writing successive multiplications. Rigorously, when written $x_1\ldots x_n$ for $n \geq 2$, it is defined inductively on $n$ via specifying the result to be $(x_1\ldots x_{n-1})x_n$. The convention is that for $n = 0$ this is simply the identity.

    In particular one can unambiguously write out the power of an element:
    \[
        a^n := \begin{cases}
            \underbrace{a\ldots a}_{n}  & n > 0 \\
            e                           & n = 0 \\
            \underbrace{a\ldots a}_{-n} & n < 0 \\
        \end{cases}
    \]
    This gives $a^m \cdot a^n = a^{m + n}$ for all $m, n \in \Z$. The cases where $m$ and $n$ are of the same sign are clear; and for those of opposite sign, applying the same elimination process as Remark \ref{rmk: group immediate results} 3) gives the desired result.

    If $G$ is abelian, in additive notation we often denote $n \cdot a := a^n$.
\end{remark}

\begin{definition}
    If $G$ and $H$ are groups, a \textbf{group homomorphism} $f: G \to H$ is a map s.t. $f(a\cdot b) = f(a) \cdot f(b)$ for all $a, b \in G$. 
\end{definition}

\begin{proposition}\label{prop: grp homo preserve identity and inverse}
    If $f: G \to H$ is a group homomorphism, then $f(e_G) = e_H$, and $f(a^{-1}) = (f(a))^{-1}$.
\end{proposition}

\begin{proof}
    By Remark \ref{rmk: group immediate results} 4) and the property of identity, we have
    \[
        f(e_G) \cdot e_H = f(e_G) = f(e_G \cdot e_G) = f(e_G) \cdot f(e_G) \implies e_H = f(e_G)
    \]
    For the second statement, use the above result:
    \[
        e_H = f(e_G) = f(a \cdot a^{-1}) = f(a) \cdot f(a^{-1})
    \]
    By the definition $(f(a))^{-1}$ is the inverse of $f(a)$. By the uniqueness of inverse this gives $f(a^{-1}) = f(a^{-1})$.
\end{proof}

\begin{remark}
    Given $f: G \to H$, $g: H \to K$ which are both $f$ and $g$ are group homomorphisms, then $f \circ g$ is also a group homomorphism. This results from the fact that
    \[
        f(g(a \cdot b)) = f(g(a) \cdot g(b)) = f(g(a)) \cdot f(g(b))
    \]
    The fact that morphism is closed w.r.t. composition implies that the groups form a category \underline{Grps}.
\end{remark}

\begin{definition}
    If $G$ and $H$ are groups, then $f: G \to H$ is a \textbf{group isomorphism} if it is a bijective group homomorphism.
\end{definition}

\begin{proposition}\label{prop: categorical def of group isomorphism}
    $f: G \to H$ being a group homomorphism is a group isomorphism if and only if there exists a group homomorphism $g: H \to G$ s.t. $g \circ f = \Id_G$, and $f \circ g = \Id_H$.
\end{proposition}

\begin{proof}
    It suffices to show implication in two directions:
    \begin{enumerate}
        \item[$\Rightarrow$:] Since $f$ is bijective, there must admit a (pointwise) inverse of $f$ s.t. $f^{-1} \circ f = \Id_{G}$, $f \circ f^{-1} = \Id_H$. Define $g = f^{-1}$. It suffices to check that $g$ is a group homomorphism. To prove this we need to verify that for all $u, v \in H$, $g(u \cdot v) = g(u) \cdot g(v)$. Since $f$ is bijective, $f$ is in particular injective, i.e. $a = b$ if and only if $f(a) = f(b)$ for all $a, b \in G$. Therefore to verify the equality above it suffices to verify the equality after applying $f$, i.e. $f \circ g(u \cdot v) = f \circ g(u) \cdot f \circ g(v)$. Then the equality holds as $f \circ g = \Id_H$. 
        \item[$\Leftarrow$:] Prove the contrapositive. If $f$ is not injective, then $g$ cannot be well-defined; and if $f$ is not surjective, then the domain of the composition $f \circ g$ is not the whole $H$.
    \end{enumerate}
\end{proof}

\begin{remark}
    Recall that under the context of categories, isomorphisms are defined as in Proposition \ref{prop: categorical def of group isomorphism}. The same proposition implies that group isomorphisms are isomorphisms in the categorical sense. 
\end{remark}

\begin{remark}
    If there exists an isomorphism $f: G \to H$ between groups $G$ and $H$, then $G$ and $H$ are considered as \textbf{isomorphic}, denoted $G \cong H$. This is an equivalence relation as compositions of isomorphisms are still isomorphisms.
\end{remark}

\begin{definition}
    Let $G$ be a group. Then a \textbf{subgroup} of $G$ is a subset $H \subseteq G$, which is in it self a group; and the inclusion map $i: H \hookrightarrow G$ is a group homomorphism. $H$ being the subgroup of $G$ is denoted as $H \leq G$.
\end{definition}

\begin{remark}
    The fact that the inclusion map is required to be a group homomorphism implies that the operation in $H$ is simply the restriction of the operation in $G$.
\end{remark}

\begin{proposition}\label{prop: subgroup test}
    Let $G$ be a group, and $H \subseteq G$ a subset. Then the followings are equivalent:
    \begin{enumerate}[label=\roman*)]
        \item $H$ is a subgroup of $G$.
        \item The following three conditions are satisfied:
        \begin{enumerate}[label=\arabic*)]
            \item For all $a, b \in H$, $a \cdot b \in H$.
            \item $e_G \in H$.
            \item (Under the same operation of $G$) $a^{-1} \in H$ for all $a \in H$. 
        \end{enumerate}
        \item $H$ is nonempty; and for all $x, y \in H$, $x \cdot y^{-1} \in H$.
    \end{enumerate}
    The third condition is often used to test whether $H \subseteq G$ gives a subgroup. 
\end{proposition}

\begin{proof}
    Verify the following implications:
    \begin{itemize}
        \item i) \implies ii). By the definition of subgroup, $H$ together with the same operation is a group, which by the definition of group is closed w.r.t. the group; and every element should admit an inverse. By the fact that $i$ is an inclusion, and by Proposition \ref{prop: grp homo preserve identity and inverse} $i(e_H) = e_G$ with $e_G = e_H$. 
        \item ii) \implies i). Check that $H$ is a group: associativity is given by the fact that the operation is identical to that in $G$. and $G$ is a group; existence of inverse and identity results directly from hypothesis 2) and 3); and the operation is defined as $H \times H \to H$ given by hypothesis 1).  
        \item ii) \implies iii). By 2) $H$ is nonempty. For all $x, y \in H$, by 3) $y^{-1} \in H$; and by 1) $x \cdot y^{-1} \in H$ given that both $x$ and $y^{-1}$ are in $H$. 
        \item iii) \implies ii). Since $H$ is nonempty, there exists $a \in H$. iii) implies that $a \cdot a^{-1} = e_G \in H$, giving 2). For all $a \in H$, let $x = e_G$ and $y = a$, which gives $a^{-1} \in H$, satisfying 3). For all $a, b \in H$, letting $x = a, y = b^{-1}$ gives $a \cdot b \in H$.  
    \end{itemize}
\end{proof}

\begin{proposition}
    Let $f: G \to H$ be a group homomorphism, then if $G' \leq G$, then $f(G') \leq H$.
\end{proposition}

\begin{proof}
    Apply the result of Proposition \ref{prop: subgroup test}. Since $G' \leq G$, $e_G \in G'$, and bby Proposition \ref{prop: grp homo preserve identity and inverse}, $f(e_G) = e_H$, giving that $f(G')$ is nonempty. For all $x, y \in f(G')$, let $u, v \in G'$ s.t. $x = f(u), y = f(v)$. Since $G'$ is a subgroup of $G$, $u\cdot v^{-1} \in G'$. By Proposition \ref{prop: grp homo preserve identity and inverse}, this implies $f(u) \cdot f(v^{-1}) = f(u) \cdot f(v^{-1}) \in f(G')$, which gives that $f(G') \leq H$.
\end{proof}

\begin{proposition}\label{prop: kernel gives a subgroup}
    Let $f: G \to H$ be a group homomorphism. If $H' \leq H$, then $f^{-1} (H') \leq G$. In particular, $f^{-1}(e_H) = \ker f := \{u \in G \mid f(u) = e_H\}$ is a subgroup of $G$.
\end{proposition}

\begin{proof}
    Apply the same argument as in the above proposition. $H' \leq H \implies e_H \in H' \implies e_G \in f^{-1}(H')$, i.e. $f^{-1}(H')$ is nonempty. For all $u, v \in f^{-1}(H')$, $f(u\cdot v^{-1}) = f(u) f(v)^{-1} \in H'$ since $H' \leq H$, which implies that $u \cdot v^{-1} \in f^{-1}(H')$, i.e. $f^{-1}(H')$ is a group. 
\end{proof}

\begin{proposition}
    Let $f: G \to H$ be a group homomorphism. Then $f$ is injective if and only if $\ker f = \{e_G\}$.
\end{proposition}

\begin{proof}
    Proceed by showing implication in both directions:
    \begin{enumerate}
        \item[$\Rightarrow$:] Let $u \in \ker f$. Then $f(a) = f(a) \cdot e = f(a) \cdot f(u) = f(a \cdot u)$. But $f$ being injective implies that $a = a\cdot u$, i.e. $u = e$.
        \item[$\Leftarrow$:] For $u, v \in G$ s.t. $f(u) = f(v)$, we have $e = f(u) \cdot (f(v))^{-1} = f(u) \cdot f(v^{-1}) = f(u \cdot v^{-1}) \implies that u \cdot v^{-1} \in \ker f$. But since the only element in $\ker f$ is the identity, this gives $u \cdot v^{-1} = e \implies u = v$, i.e. $f$ is injective.
    \end{enumerate}
\end{proof}

\section{Group of Permutations}

\begin{definition}
    Given a set $\Omega$, the \textbf{permutation group} is defined to be $S_{\Omega} := \{f: \Omega \to \Omega \mid f\ \text{bijection}\}$. Since compositions of bijective maps are still bijective, defining the operation to be composition gives this a group structure.
\end{definition}

\begin{remark}
    Notice that the permutation group structure depends only on the cardinality of the group on which permutations are considered. Explicitly, for $\alpha: \Omega \to \Omega'$ a bijection, there exists an isomorphism between the corresponding groups of permutations: $\beta: S_{\Omega} \to S_{\Omega'}: f \mapsto \alpha \circ f \circ \alpha^{-1}$. This is indeed an isomorphism as this is first a group homomorphism since
    \[
        \beta(f \circ g) = \alpha \circ f \circ g \circ \alpha^{-1} = \alpha \circ f \circ (\alpha^{-1} \circ) \alpha \circ g \circ \alpha^{-1} = \beta(f) \circ \beta(g)
    \]
    and this being an isomorphism follows from the fact that there exists an obvious inverse $\beta^{-1}: f \mapsto \alpha^{-1} \circ f \circ \alpha$. Therefore it suffices to denote such permutation group by the cardinality of $\Omega$: for $\Omega = \{1, \dots, n\}$ $S_{\Omega}$ is denoted as $S_n$.
\end{remark}

\begin{proposition}[Cayley]
    Every group can be embedded into some $S_{\Omega}$. Explicitly, for group $G$ the map $\alpha: G \to S_G$ s.t. $g \mapsto \alpha_g$ where $\alpha_g(h) = gh$ ($\alpha_g$ is the action of $G$ on $G$ defined by multiplication by $g$.) is an injective group homomorphism.
\end{proposition}

\begin{proof}
    It suffices to syntactically check that the following requirements are satisfied:
    \begin{itemize}
        \item \emph{$\alpha_g \in S_G$.} It suffices to check that indeed multiplication by an element in the group gives a bijection. This is clear as the action has an inverse, namely multiplying the inverse of that element. 
        \item \emph{$\alpha$ gives a group homomorphism.} By definition $\alpha_{gh} = \alpha_g \cdot \alpha_h$.
        \item \emph{$\alpha$ is injective.} It suffices to check that $\ker \alpha = e_G$. This is indeed the case, as for $g \in G$ s.t. $\alpha_g = \Id$, $\alpha_g(e_G) = g \cdot e_G = e_G \implies g = e_G$.
    \end{itemize}
\end{proof}

\section{Groups Generated by a Subset}

\begin{remark}
    If $(H_i)_{i \in I}$ is a family of subgroups of $G$, then $\bigcap_{i \in I} H_i$ is also a subgroup of $G$. This can be verified by taking an element in the intersection, and check each rule of group is satisfied in each of the $H_i$s.
\end{remark}

\begin{definition}
    If $A \subseteq G$ is a subset of $G$, then the \textbf{subgroup generated by $A$} is defined as
    \[
        \inner{A} := \bigcap_{A \subseteq H \leq G} H
    \]
\end{definition}

\begin{remark}
    By definition $\inner{A}$ is well-defined as it is described by concrete elements in the group; and as in particular $A \subseteq G \leq G$. By the previous remark, $\inner{A}$ is a subgroup of $G$. It is also the smallest subgroup that contains $A$.
\end{remark}

\begin{proposition}\label{prop: explicit presentation of generated subgroup}
    Let $A \subseteq G$ be a subset of $G$, then $\inner{A} = \left\{ x_1\dots x_n \mid n \in \Z_{>0}; \forall i, x_i \in G \text{ or } x_i^{-1} \in G \right\}$. For $n = 0$, define $x_1 \dots x_n = e$. 
\end{proposition}

\begin{proof}
    Proceed by double inclusion:
    \begin{enumerate}
        \item[$\subseteq$:] Proceed to show that RHS satisfies the definition of the $H$s above. For RHS consider $n = 1$, with $x_1 \in G$ which takes all elements in $G$. This gives $A \subseteq$ RHS. Further use Proposition \ref{prop: subgroup test}, which for any $x_1\dots x_m, y_1\dots y_n \in$ RHS, each summand of $x_1 \dots x_m (y_1 \dots y_n)^{-1} = x_1 \dots x_m y_n^{-1} \dots y_1^{-1}$ is either in $A$ or its inverse is in $A$ implying that RHS is a group. Definition above gives the subset relation.
        \item[$\supseteq$:] It suffices to verify that any element in the specified form is in $\inner{A}$. This is the case as for $x_1\dots x_n$ where for all $i$, either $x_i \in A$ or $-x_i \in A$, $x_i \in \inner{A}$ by definition, and multiplication of two elements in the group is still in the group by closure of the operation.  
    \end{enumerate}
\end{proof}

\begin{definition}
    The following defines some common terminology for characterization of a group:
    \begin{itemize}
        \item $G$ is \textbf{finitely generated} if there exists a finite set $A \subseteq G$ s.t. $G = \inner{A}$.
        \item $G$ is \textbf{finite} if it has finitely many elements. 
        \item The \textbf{order} of $G$, denoted $\abs{G}$, is the number of elements in $G$ if it is finite; or $\infty$ if $G$ is not finite (infinite).
        \item $G$ is \textbf{cyclic} if it attains a generating set with a single element $a$. In this case $G$ is denoted as $G = \inner{a}$.
        \item The \textbf{order} of $a \in G$, denoted $\abs{a}$ is the order of $\inner{a}$.  
    \end{itemize}
\end{definition}

\begin{remark}
    Cyclic groups are abelian. By the alternative definition provided in Proposition \ref{prop: explicit presentation of generated subgroup}, $\inner{a} = \{ a^m \mid m \in \Z \}$.
\end{remark}

\begin{proposition}\label{prop: characterization of cyclic group}
    A group $G$ is cyclic if and only if $G \simeq \Z$ if $G$ is infinite, or $G \simeq \Z/n\Z$ for some $n \in \Z_{>0}$.  
\end{proposition}

\begin{proof}
    Choose $a \in G$ s.t. $G = \inner{a}$. Proceed via showing implication in both directions:
    \begin{enumerate}
        \item[$\Rightarrow$:] Consider $f: \Z \to G$ s.t. $f(1) = a$. This is a group homomorphism,  Then either
        \begin{itemize}
            \item \emph{$f$ is injective.} By definition of cyclic groups, for any $s \in G$ there exists $m \in G$ s.t. $s = a^m$. Then $f(m) = s$ according to the definition of $f$, giving that $f$ is surjective. Then this falls into the first case, giving $G \simeq \Z$.
            \item \emph{$f$ is not injective.} Then there are nonzero elements in $f$. Since $\ker f \subseteq \Z$, there exists a smallest positive element. Define the map $f_n: \Z/n\Z \to G$ s.t. $[1] \mapsto a$. Check the followings:
            \begin{itemize}
                \item \emph{$f_n$ is well-defined.} It suffices to check that if $[m_1] = [m_2]$, then $f([m_1]) = (f[m_2])$. This is indeed the case as
                \[
                    f([m_1]) = a^{m_1} \overset{!}{=} a^{m_1} \cdot a^{(m_2 - m_1)} = a^{m_2} \cdot a^{nk} = a^{m_2} \cdot (a^n)^k = a^{m_2} = f([m_2])
                \]
                for some $k \in \Z$, where $\overset{!}{=}$ holds since $[m_1] = [m_2]$ implies $n \mid (m_1 - m_2)$. This gives $a^{m_1 - m_2} = e$ since $a^n = e$. 
                \item \emph{$f_n$ is injective.} For $a \in \Z$ s.t. $f_n([a]) = 0$, $a = 0$ as otherwise this conflicts with the hypothesis that $n$ is the smallest of such integers. 
                \item \emph{$f_n$ is surjective.} Follows from the same argument in the case where $G$ is infinite.
            \end{itemize}
        \end{itemize}
        \item[$\Leftarrow$:] Since $\Z = \inner{1}$ and $\Z/n\Z = \inner{[1]}$, both of which are cyclic. 
    \end{enumerate}
\end{proof}

\section{The Dihedral Group}

\begin{definition}
    Let $n \geq 3$, and $P_n \subset \R^2 \simeq \C$ be the regular $n$-gon s.t. its vertices are at the $n$-th roots of 1. Then the \textbf{dihedral group} $D_{2n}$ is the group of symmetry of $P_n$. Alternatively, one can write
    \[
        D_{2n} = \{ \varphi \in \GL_2(\R) \mid \varphi(P_n) = P_n \}
    \]
\end{definition}

\begin{remark}
    We have a injective map $\alpha: D_{2n} \to S_n$, where $\alpha(\varphi)$ is given by the restriction of $\varphi$ to the vertices of $P_n$. This map is injective as $\{v_1, \dots, v_n\}$ spans $\R^2$. Therefore, specifying how the vertices are transformed (permuted) fixes the whole linear transformation.
\end{remark}

\begin{remark} \label{rmk: rule of computing dihedral group}
    Notice the following relations: by definition of rotation $\sigma^n = e$; and $\sigma \tau \sigma = \tau$, which implies $\sigma^{n-1} \tau = \tau \sigma$. This enables changing the sequence of applying $\sigma$s and $\tau$s.
\end{remark}

\begin{proposition}
    For a fixed $n$, let $\sigma$ be the operation of counter-clockwise rotation by $\frac{2\pi}{n}$ on $P_n$; and $\tau_j$ be the operation of symmetry w.r.t. the symmetry axis passing through the vertex $j$ (which is a direction; invariant w.r.t. transformations on $P_n$). Then for every $\alpha \in D_{2n}$, it must be in the form of $\sigma^i$ or $\sigma^i \cdot \tau_j$, for some $i, j \in \Z$.
\end{proposition}

\begin{proof}
    How the operations permute the vertices is characterized by
    \[
        \sigma: v_k \mapsto v_{k + 1} \qquad \tau: v_{j+k} \mapsto v_{j-k}
    \]
    Following the strategy of the previous remark, to fix the whole operation $\alpha$ it suffices to fix how vertices are transformed. Since elements of $D_{2n}$ are linear transformations, they map line segments to line segments, and therefore adjacent vertices to adjacent vertices. Then for $v_1 \mapsto v_{i+1}$, either $v_2 \mapsto v_{i+2}$, then $\alpha = \sigma^i$; or $v_2 \mapsto v_{i}$, then $\alpha = \sigma^i \tau_j$. The indices are considered modulo $n$ and then plus 1. 
\end{proof}

\begin{remark}
    Using Remark \ref{rmk: rule of computing dihedral group}, we can check that indeed $\inner{D_{2n}} = D_{2n}$, by applying the remark to move all the rotations to the left of symmetries, and the reduce the expression by relations $\sigma^n = \tau^2 = e$. 
\end{remark}

\section{Product of Groups}

\begin{definition}[Product of Groups]
    Suppose that we have a family of groups $(G_i)_{i \in I}$. The \textbf{product} of groups is defined as
    \[
        \Pi_{i \in I} G_i := \left\{ (x_i)_{i \in I} \mid x_i \in G_i \forall i \in I \right\}
    \]
    with the operation defined component-wise i.e. $(x_i)_{i \in I} \cdot (y_i)_{i \in I} := (x_i y_i)_{i \in I}$.
\end{definition}

\begin{remark}
    By the definition of the operation, the identity in the product of groups $(G_i)_{i \in I}$ is $(e_i)_{i \in I}$ where $e_i$ is the unique identity element in $G_i$; and the inverse of $(x_i)_{i \in I}$ is $(x_i^{-1})_{i \in I}$.
\end{remark}

\begin{proposition}[Universal Property of Product of Groups]\label{prop: universal property of product of groups}
    Let group homomorphism $\pi_j: \Pi_{i \in I} G_i \to G_j, (x_i)_{i \in I} \mapsto x_j$ be the projections. Then given group homomorphisms $f_i: H \to G_i$ for all $i$, there exists a unique group homomorphism $f: H \to \Pi_{i \in I} G_i$ s.t. $\pi_i \circ f = f_i$ for all $i \in I$, i.e. the following diagram commute:

    \begin{minipage}{\linewidth}
        \centering
        \begin{tikzcd}
            H \arrow[rr, dashed, "f"] \arrow[rrdd, "f_j"] & & \Pi_{i \in I} G_i \arrow[dd, "\pi_j"]\\
            & & \\
            & & G_j
        \end{tikzcd}
    \end{minipage}
\end{proposition}

\begin{proof}
    Since the diagram is required to commute, the homomorphism $f$ can be only defined as $f(x) = (f_i(x))_{i \in I}$, which gives the uniqueness. Existence follows from the fact that $f_i$s are group homomorphisms for all $i$, which implies that $f$ is also a group homomorphism.
\end{proof}

\begin{example}[Chinese Remainder Theorem]
    Let $m, n \in \Z_{\geq 0}$ which are relatively prime. Then there exists group isomorphism $\Z/mn\Z \simeq \Z/m\Z \times \Z/n\Z$.
\end{example}

\begin{proof}
    Consider group homomorphisms: 
    \[
        f: \Z/mn\Z \to \Z/m\Z, \quad [x + mn\Z] \mapsto [x + m\Z]
    \]
    \[
        g: \Z/mn\Z \to \Z/n\Z, \quad [x + mn\Z] \mapsto [x + n\Z]
    \]
    Check that $f$ and $g$ are well-defined. For $f$, let $a = [x + mn\Z] = b = [y + mn\Z]$. This implies that $mn \mid (x - y)$. By definition, $f(a) = [x + m\Z], f(b) = [y + m\Z]$. But this implies that $[x + m\Z] = [y + m\Z]$ as $mn \mid (x - y) \implies m \mid (x - y)$. The well-definedness of $g$ is similar.

    Use the universal property above (Proposition \ref{prop: universal property of product of groups}), there exists a unique $h: \Z/mn\Z \to \Z/m\Z \times \Z/n\Z$ s.t. $h_1 = f, h_2 = g$ where $h_i$ indicates the projection to $i$-th field after applying $h$. Check that this is an isomorphism:
    \begin{itemize}
        \item $h$ is injective. Consider the kernel of $f$: for all $[x + mn\Z] \in \ker f$, $[x + m\Z] = 0$ and $[x + n\Z] = 0$. But this implies that $m \mid x$ and $n \mid x$, i.e. $mn \mid x$, which gives $[x + mn\Z] = 0$. That is, elements in $\ker f$ are identically zero, which gives the injectivity.
        \item Notice that $\Z/mn\Z$ has $mn$ elements, while $\Z/m \times \Z/n\Z$ has $m \cdot n = mn$ elements. Therefore $h$ being injective implies $h$ being bijective. 
    \end{itemize}
\end{proof}

\section{Congruence Relations}

\begin{definition}[Left/Right Congruence]
    Let $G$ be a group, with $H \leq G$. Then for $x, y \in G$,
    \begin{itemize}
        \item $x$ and $y$ are \textbf{left congruent} mod $H$, denoted $x \equiv_{\ell} y \mod H$ if $x^{-1}y \in H$.
        \item $x$ and $y$ are \textbf{right congruent} mod $H$, denoted $x \equiv_{r} y \mod H$ if $xy^{-1} \in H$.
    \end{itemize}
\end{definition}

\begin{remark}
    $\equiv_{\ell}$ and $\equiv_r$ are equivalence relations. The equivalence classes are noted as $xH$ and $Hx$ for $x \in G$, respectively.
\end{remark}

\begin{notation}
    If $G$ is abelian, the operation is written additively. The congruence classes will then be denoted as $x + H$ and $H + x$ for left and right congruence classes, respectively.
\end{notation}

\begin{proof}
    The proof is similar for two equivalence relations, so we only check for left congruence:
    \begin{itemize}
        \item \emph{$\equiv_{\ell}$ is Reflexive.} $x^{-1} \cdot x = e \in H$.
        \item \emph{$\equiv_{\ell}$ is symmetric.} If $x^{-1}y \in H$, given that $H$ is a subgroup of $G$, $(x^{-1}y)^{-1} \in H$. This implies that $y^{-1}x \in H$, i.e. $y \equiv_{\ell} x \mod H$.
        \item \emph{$\equiv_{\ell}$ is transitive.} Suppose that $x \equiv_{\ell} y \mod H, y \equiv_{\ell} z \mod H$. By the fact that subgroups are closed, $(x^{-1}y)(y^{-1}z) = x^{-1}z \in H$.
    \end{itemize}
\end{proof}

\begin{remark}\label{rmk: group as disjoint union of cong classes}
    $G$ is the disjoint union of equivalence classes w.r.t. $\equiv_{\ell}$. For $x, y \in G$ s.t. $x \equiv_{\ell} y \mod H$, there exists $h \in H$ s.t. $x = yh$.
\end{remark}

\begin{proposition}
    There is a bijection between $xH$ and $Hx$ for all $x \in G, H \leq G$. 
\end{proposition}

\begin{proof}
    Define the map $\varphi: \{xH \mid x \in G\} \to \{Hx \mid x \in G\}$, $gH \mapsto Hg^{-1}$. Check that this is well-defined: for $g_1, g_2 \in G$ s.t. $g_1 H = g_2 H$, there exists $h \in H$ s.t. $g_1 = g_2 h$. Then $Hg_1 = H(g_2 h)^{-1} = Hh^{-1} g_2^{-1} = Hg_2^{-1}$. It has inverse $Hg \mapsto g^{-1}H$, with well-definedness similarly proved, which implies that $\varphi$ is a bijection.
\end{proof}

\begin{remark}
    In the prove above, we cannot define $\varphi: gH \mapsto Hg$ as in this case this is not well-defined. Specifically, if $g_1$ does not commute with $h$ for $g_1 = g_2 h$, $\varphi(g_2 H) = Hg_1 h$ which is not necessarily equal to $H g_1$.
\end{remark}

Since the number of congruence classes w.r.t. $x \in G$ does not change with choice of left or right congruence classes and depends only on $H$, the following definition is well-defined:

\begin{definition}[Index]
    Let $G$ be a group, with $H \leq G$. Then the number of distinct $xH$ for $x \in G$ is the \textbf{index} of $H$ in $G$, denoted as $(G : H)$.
\end{definition}

\begin{remark}
    For all $g_1, g_2 \in H$, there exists bijections $g_1 H \mapsto g_2 H$ and $H g_1 \mapsto g_2 H$, given by multiplication on the left by $g_2 g_1^{-1}$, and multiplication on the right by $g_1^{-1} g_2$, respectively. 
\end{remark}

\begin{theorem}[Lagrange]
    Let $G$ be a group. If $H \leq G$, and $G$ is finite, then $\abs{G} = \abs{H} \cdot (G : H)$.
\end{theorem}

\begin{proof}
    By Remark \ref{rmk: group as disjoint union of cong classes}, $G$ is the disjoint union of congruence classes. There are $(G : H)$ congruence classes (in the form of $xH$ for $x \in G \smallsetminus H$), with each having $\abs{H}$ elements (given by $\{ xh \mid h \in H \}$).
\end{proof}

\begin{corollary}
    In particular, for all $H \leq G$, $\abs{H} \mid \abs{G}$. If $G$ is finite, for all $g \in G$, $\abs{\inner{g}} \mid \abs{G}$, i.e. $g^{\abs{G}} = g^{\abs{\inner{g}} \cdot (G : \inner{g})} = e$.
\end{corollary}

\begin{example}[Fermat's Little Theorem]
    Let $G = (\Z/p\Z)^{\times}$ with $p$ prime. Then $\abs{G} = p - 1$. For $a \in \Z$ s.t. $p \nmid a$, $\abs{[a]} = p-1$, which implies that $a^{p-1} \equiv 1 \mod p$. 
\end{example}

We now seek to define a group structure on the congruence classes modulo a subgroup $H \leq G$. The issue is that the operation is not necessarily well-defined. The natural definition of the group operation is given via $(g_1 H, g_2 H) \mapsto (g_1 g_2 H)$. For $g_1 \equiv_{\ell} g_1' \mod H, g_2 \equiv_{\ell} g_2' \mod H$ we would like $g_1 g_2 \equiv_{\ell} g_1' g_2'$. In terms of the elements, we have $g_1 g_1'^{-1} g_2 g_2'^{-1} \in H$ and we want $g_1 g_2 g_2'^{-1} g_1^{-1} \in H$. This requires extra requirements on $H$.

\begin{claim}
    The following two conditions are equivalent:
    \begin{itemize}
        \item For all $g_1^{-1}g_1' \in H, g_2^{-1}g_2' \in H$, this implies $(g_1 g_2)^{-1}(g_1 g_2)' \in H$.
        \item For all $x \in G, h \in H, xhx^{-1} \in H$.
    \end{itemize}
\end{claim}

\begin{proof}
    Consider the following constructions in two directions:
    \begin{itemize}
        \item[$\Rightarrow$] Notice $g_1^{-1} g_1 \in H$ by hypothesis. Choose $g_2^{-1} = x, g_2' = x^{-1}$.
        \item[$\Leftarrow$] Notice $(g_1 g_2)^{-1}(g_1 g_2)' = g_2^{-1} g_1^{-1} g_1' g_2' \in H$. Choose $g_2 = g_2' = x$, with $g_1^{-1} g_1' = h$. Such $g_1$ and $g_1'$ exists by first arbitrarily choose $g_1 \in H$ then compute $g_1' = g_1 h$.
    \end{itemize}
\end{proof}

This gives rise to the definition of normal subgroups, and the formulation quotient with respect to it, as follows.

\section{Normal Subgroup, Quotient Group and Isomorphism Theorems}

\begin{definition}[Normal Subgroup]
    A subgroup $H \leq G$ is \textbf{normal} if for all $x \in G, xHx^{-1} \in H$, where
    \[
        xHx^{-1} := \{ xhx^{-1} \mid h \in H \}
    \]
    Normal subgroups are denoted by $H \normalin G$.
\end{definition}

\begin{definition}[Quotient Group]
    Let $G$ be a group, and $H \normalin G$. Then the \textbf{quotient group} $G/H$ is the set of left equivalence classes w.r.t. $H$, together with the group operation $(g_1 H)(g_2 H) := (g_1 g_2) H$.
\end{definition}

\begin{remark}\label{rmk: projection to quotient}
    Explicitly check that this gives a group structure: by definition we have the identity element $eH$, with the inverse of $g_1 H = (g_1^{-1})H$. The well-definedness of the group follows from the fact that all the left congruence classes of $H$ are well-defined, i.e. operations on it does not depends on the choice if representative. This also gives a group homomorphism $\pi: G \to G/H$ with $x \mapsto xH$. This is indeed a group homomorphism as $\pi(ab) = (ab) H = aH bH = \pi(a) \pi(b)$. 
\end{remark}
    
\begin{remark}
    The definition above is identical when formulated in terms of left or right congruence classes. Since we have the bijection between left and right congruence classes, to check that the definitions are identical it suffices to check that the bijection is compatible with the group operation specified. This indeed can be defined as such, as denoting the bijection to be $\Phi: xH \mapsto Hx^{-1}$ we have
    \[
        \Phi(xH \cdot yH) = Hx^{-1} \cdot Hy^{-1} := H y^{-1} x^{-1} = \Phi((xy) H)
    \]
\end{remark}

\begin{example}
    The followings give some examples of normal subgroups:
    \begin{enumerate}
        \item Trivially, $\{e\}$ and $G$ are normal subgroups of $G$.
        \item If $G$ is abelian, for all $x \in G, H \leq G$, we have $xHx^{-1} = xx^{-1}H = H$ which implies that every subgroup is normal. Further the quotient $G/H$ is abelian, as by Remark \ref{rmk: projection to quotient}, the operation in $G$ induces the operation in $G/H$.
        \item Consider the nontrivial case, where $G = D_3 = \inner{\sigma, \tau} = \{e, \sigma, \sigma^2, \tau, \tau\sigma, \tau\sigma^2\}$. Then
            \begin{itemize}
                \item Consider $H_1 = \inner{\sigma} = \{e, \sigma, \sigma^2\}$. Check $\tau\sigma\tau^{-1} = \tau\sigma\tau = \sigma^2\tau\tau = \sigma^2 \in H_1$; and $\tau\sigma^2\tau^{-1} = \tau\sigma^2\tau = \sigma\tau\tau = \sigma \in H_1$. Similarly for $\sigma\tau$ and $\sigma^2\tau$. This implies that $H_1$ is normal in $G$.
                \item Consider $H_2 = \{ e, \tau \}$. we have $\sigma\tau\sigma^{-1} = \sigma\tau\sigma^2 = \tau\sigma = \sigma^2 \tau \notin H_2$ which implies that $H_2$ is not a normal subgroup.
            \end{itemize}
    \end{enumerate}
\end{example}

\begin{proposition}\label{prop: equivalence definition of normal subgroup}
    If $H \leq G$, then the following statements are equivalent:
    \begin{enumerate}[label=\arabic*)]
        \item $H$ is a normal subgroup of $G$.
        \item $gH = Hg$ for all $g \in G$, i.e. the left and right equivalence classes are equal.
        \item $gHg^{-1} = H$ for all $g \in G$.
    \end{enumerate}
\end{proposition}

\begin{proof}
    First see that statement 2) and 3) are equivalent, by right multiplying $g$ and $g^{-1}$, respectively. For the rest of the equivalence, consider
    \begin{itemize}
        \item \emph{3) \implies 1)}. This in particular implies that $xhx^{-1} \in H$ for all $h \in H$, which is exactly the definition of normal subgroups.
        \item \emph{1) \implies 3)}. The definition of normal subgroups implies that $gHg^{-1} \subseteq H$ for all $g \in G$. Apply this to $g^{-1} \in G$ gives $g^{-1} H g \subseteq H \implies H \subseteq gHg^{-1}$. Combining the two statements gives the desired equality. Alternatively, one can see that conjugating by $g$ is an isomorphism onto its image, where inclusion in one side implies that this is bijective. 
    \end{itemize}
\end{proof}

\begin{corollary}
    In Every subgroup with index 2 is normal.
\end{corollary}

\begin{proof}
    Let $H \leq G$ be index 2. Then the left congruence classes are given by $\{ H, gH \}$ for $g \in G \smallsetminus H$; with the right equivalence classes $\{ H, Hg \}$. This implies that $gH = Hg$ in terms of individual elements. By Proposition \ref{prop: equivalence definition of normal subgroup} this implies that $H$ is normal in $G$. 
\end{proof}

\begin{proposition}\label{prop: kernel gives a normal subgroup}
    Let $H \subseteq G$ be a subset. Then $H$ is a normal subgroup in $G$ if and only if there is some group homomorphism $f: G \to G'$ s.t. $\ker f = H$.
\end{proposition}

\begin{proof}
    Consider implication in two directions:
    \begin{enumerate}
        \item[$\Rightarrow$] Consider the group homomorphism induced by the quotient structure: $\pi: G \to G/H$, $g \mapsto gH$. Then $\ker \pi = \{ g \in G \mid gH = H \}$. This implies that $g \in H$.
        \item[$\Leftarrow$] By Proposition \ref{prop: kernel gives a subgroup} $H$ is a subgroup in $G$. Check that it is normal: for all $h \in H$, $g \in G$, we have
        \[
            f(ghg^{-1}) = f(g) f(h) f(g^{-1}) = f(g) (f(g))^{-1} = e \implies ghg^{-1} \in H
        \]
    \end{enumerate}
\end{proof}

\begin{proposition}[Universal Property of Quotient Group]\label{prop: universal property of quotient group}
    Let $G$ be a group, and $H$ is normal in $G$. Let $\pi: G \to G/H$, and $f: G \to G'$ be group homomorphisms s.t. $H \subseteq \ker f$. Then there exists a unique group homomorphism $\bar{f} : G/H \to G'$ s.t. $\bar{f} \circ \pi = f$, i.e. the following diagram commutes:

    \begin{minipage}{\linewidth}
        \centering
        \begin{tikzcd}
            G \arrow[rr, "\pi"] \arrow[rrdd, "f"] & & G/H \arrow[dd, dashed, "\bar{f}"] \\
            & & \\
            & & G'
        \end{tikzcd}
    \end{minipage}
\end{proposition}

\begin{proof}
    For uniqueness, notice that since the diagram is required to commute, we have $\bar{f} (gH) = f(g)$ for all $g \in G$. Since $\pi$ is surjective, the behavior of $\bar{f}$ is described only on image of $\pi$, i.e. on congruence classes of form $gH$ for $g \in G$. This gives the uniqueness of the map.
    
    For existence, check that $f$ is well-defined, and is indeed a group homomorphism:
    \begin{itemize}
        \item \emph{$\bar{f}$ is well-defined.} For $gH = g'H$, we want to show that $\bar{f}(gH) = \bar{f}(g'H)$, i.e. $f(g) = f(g')$. But $gH = g'H$ implies $g^{-1}g' \in H$, i.e. $f(g) \cdot (f(g'))^{-1} = f(g\cdot {g'}^{-1}) \in f(H) = e$, which gives $f(g) = f(g')$.
        \item \emph{$\bar{f}$ is a group homomorphism.} This is simply paraphrasing of the definition $(gH)(g'H) = (gg')H$.
    \end{itemize}
\end{proof}

\begin{theorem}[First Isomorphism Theorem] \label{thm: first isomorphism theorem}
    If $f: G \to G'$ is a surjective group homomorphism, then $G' \simeq G / \ker f$, i.e. the following diagram commutes with $\bar{f}$ an isomorphism:

    \begin{minipage}{\linewidth}
        \centering
        \begin{tikzcd}
            G \arrow[rr, "\pi"] \arrow[rrdd, "f"] & & G/\ker f \arrow[dd, "\bar{f}"] \\
            & & \\
            & & G'
        \end{tikzcd}
    \end{minipage}
\end{theorem}

\begin{proof}
    Uniqueness and existence of $\bar{f}$ follows from Prop \ref{prop: universal property of quotient group}. 

    Check that $\bar{f}$ is an isomorphism. Surjectivity follows from the fact that $f$ is surjective, and the diagram is required to commute. To check that $\bar{f}$ is injective, consider $\ker \bar{f}$. For, $x \in \ker \bar{f}$, $\bar{f}(x) = f(x') = e$ for $x' \in G$ s.t. $\pi(x') = x$. But this implies that $x' \in \ker f$, i.e. $\pi(x') = x = e$.
\end{proof}

\begin{corollary}
    If $f: G \to G'$ is any group homomorphism, then $\im f \simeq G/\ker f$.
\end{corollary}

\begin{remark}
    If $f: G \to G'$ is a group homomorphism, and $H'$ is normal in $G$, then $f^{-1}(H')$ is normal in $G$.
\end{remark}

\begin{proof}
    Denote $p': G' \to G'/H'$ which is the projection into the quotient. Notice that $p' \circ f (f^{-1}(H)) = e$, i.e. $f^{-1}H = \ker (p' \circ f)$. Proposition \ref{prop: kernel gives a normal subgroup} gives that $f^{-1}(H')$ is normal. 
\end{proof}

\begin{remark}\label{rmk: transformation between quotient on normal subgroups}
    Let $H$ and $H'$ be normal in $G$ and $G'$, respectively. Let $f: G \to G'$, $p: G \to G/H$, $p': G' \to G'/H'$ be group homomorphisms s.t. $f(H) \subseteq H'$. Then there exists a unique group homomorphism $\bar{f}: G/H \to G'/H'$ s.t. the following diagram commutes:
    
    \begin{minipage}{\linewidth}
        \centering
        \begin{tikzcd}
            G \arrow[rr, "f"] \arrow[dd, "p"] & & G' \arrow[dd, "p'"] \\
            & & \\
            G/H \arrow[rr, "\bar{f}"] & & G'/H'
        \end{tikzcd}        
    \end{minipage}

    Proof is by applying universal property (Proposition \ref{prop: universal property of quotient group}) on $p$ and $p' \circ f$. It is applicable as $f(H) \subseteq H'$, i.e. $H \subseteq \ker (p' \circ f)$.
\end{remark}


\begin{parenthesis} \label{pth: quotient preserves normal subgroups}
    Let $p: G \to G/H$ be the projection into the quotient. Then if $H \leq M$, then $M$ is normal in $G$ if and only if $p(M) = M/H$ is normal in $G/H$.
\end{parenthesis}

\begin{proof}
    Show implications in both directions:
    \begin{enumerate}
        \item[$\Rightarrow$] Use Remark \ref{rmk: transformation between quotient on normal subgroups}, with $G = G'$, $H' = M$, and $f$ the identity map. By hypothesis that $H \leq M$, we have $f(H) \subseteq M$. The remark says that there exists a map $\bar{f}: G/H \to G/M$, with kernel $p(M)$ by the fact that the diagram commutes. Proposition \ref{prop: kernel gives a normal subgroup} gives the fact that $p(M)$ is normal in $G/H$.
        \item[$\Leftarrow$] Since $M/H$ is normal in $G/H$ it is valid to consider the quotient $(G/H)/(M/H)$ with the projection $p': G/H \to (G/H)/(M/H)$, which is a group homomorphism. It is then clear that $\ker (p' \circ p) = M$, i.e. $M$ is a normal subgroup by Proposition \ref{prop: kernel gives a normal subgroup}.
    \end{enumerate}
\end{proof}

\begin{theorem}[Third Isomorphism Theorem]
    Let $G$ a group, and $H, M$ subgroups in $G$ s.t. $H \leq M \leq G$. Then $(G/H)(M/H) \simeq G/M$.
\end{theorem}

\begin{proof}
    Let $p: G \to G/H$ be the projection into the quotient. Consider the group homomorphism $\alpha: G/H \to G/M$, given bu $xH \mapsto xM$. $\ker \alpha = \{ xH \mid x \in M \} = p(M)$. By Parenthesis \ref{pth: quotient preserves normal subgroups} we know that $p(M)$ is normal in $G/H$. The First Isomorphism Theorem (Theorem \ref{thm: first isomorphism theorem}) gives the desired isomorphism. 
\end{proof}

The following theorem connects the subgroups in the quotient and the subgroups in the original group:

\begin{theorem}[Correspondence]
    Let $G$ be a group, and $H$ a normal subgroup in $G$. Then we have an \emph{order-preserving} bijection:
    \[
        \Phi: \{ \text{subgroups in }G/H \} \to \{ \text{subgroups of $G$ containing $H$} \} 
    \]
    which maps normal subgroups to normal subgroups. Being \emph{order-preserving} implies that $U \subseteq V$ if and only if $\Phi(U) \subseteq \Phi(V)$.
\end{theorem}

\begin{proof}
    Define $\Phi$ as $p^{-1}$ with $p$ being the projection $G \to G/H$, as by the definition of quotient groups, we have $K \subseteq G/H \implies p^{-1}K \subseteq G$ by the fact that $p^{-1}$ is order-preserving. Further by Parenthesis \ref{pth: quotient preserves normal subgroups} we have $K \normalin G/H \implies p^{-1}K \normalin G$. The images are subgroups containing $H$, as in particular we have $p^{-1}(K) \supseteq p^{-1}(e) = H$. 

    Now check that the inverse of $\Phi$ exists; and the composition in two directions are both the identity. Check the followings:
    \begin{itemize}
        \item $p(p^{-1}(K)) = K$ for $K \leq G/H$. By definition $p(p^{-1})(K) \subseteq K$. The equality follows from the fact that $p$ is surjective.
        \item $p^{-1}(p(M)) = M$ for $M \leq G$. $p^{-1}(p(M)) \supseteq M$ is given by definition; while $g \in p^{-1}(p(M))$ implies that $gH = xH$ for $x \in M$ as $p$ is surjective. But this implies that $g = xh$ for some $h \in H$, i.e. $g \in M$.
    \end{itemize}
\end{proof}

For the formulation of the Second Isomorphism Theorem, we need to first introduce some definitions:
\begin{definition}
    Let $B \leq G$. Then the \textbf{normalizer} of $B$ in $G$ is defined as
    \[
        N_G(B) := \{ g \in G \mid gBg^{-1} \in B \}
    \]
\end{definition}

\begin{remark}
    By definition of normalizer, $B$ is normal in $G$ (the normalizer makes $B$ a normal subgroup). This is also the largest subgroup of $G$ in which $B$ is normal, as suppose that there exists a larger one, it would be included in the normalizer by definition. The normalizer exists as in particular $B$ is normal in $B$, implying that $B \subseteq N_G(B)$.
\end{remark}

\begin{notation}
    Let $A, B \leq G$ be subgroups. Denote
    \[
        AB := \{ ab \mid a \in A, b \in B \}
    \]
\end{notation}

\begin{remark}
    By definition $AB$ is not necessarily a subgroup in $G$: for $a_1b_1, a_2 b_2 \in AB$, $a_1 b_1(a_2 b_2)^{-1} = a_1 b_1 b_2^{-1} a_2^{-1}$ which is not in the form of $AB$. But if $A \subseteq N_G(B)$, this is the case as we have
    \[
        a_1 b_1 b_2^{-1} a_2^{-1} = (a_1 a_2^{-1}) (a_2 b_1 b_2^{-1} a_2^{-1})
    \]
    which gives $a_1 b_1 (a_2 b_2)^{-1} = a' b'$ for $a' = a_1 a_2^{-1}$ and $b = a_2 b_1 b_2^{-1} a_2^{-1} \in B$.
\end{remark}

\begin{theorem}[Second Isomorphism Theorem]
    Let $A$ and $B$ be subgroups of $G$. Further let $A \subseteq N_G(B)$. Then $A \cap B \normaleqin A$ and $B \normaleqin AB$; and we have the isomorphism $A/(A \cap B) \simeq AB/B$.
\end{theorem}

\begin{proof}
    Notice $A \cap B \subseteq B$ and $A \subseteq N_G(B)$. Therefore, for all $b \in A \cap B$, $a \in A$, $aba^{-1} \in A \cap B$ by closure of operation in $A$ and $B$ is normal in $A$. Further $B \normalin AB$ as $(ab) b' (ab)^{-1} = abb'b^{-1} a^{-1} \in B$ since $a \in N_G(B)$. Consider $f: A \to AB$, $a \mapsto ab$ for some fixed $b \in B$. Use the result in Remark \ref{rmk: transformation between quotient on normal subgroups} to get the following commutative diagram:

    \begin{minipage}{\linewidth}
        \centering
        \begin{tikzcd}
            A \arrow[rr, "f"] \arrow[dd] & & AB \arrow[dd] \\
            & & \\
            A/(A \cap B) \arrow[rr, "\bar{f}"] & & AB/B
        \end{tikzcd}
    \end{minipage}
    $f$ is an isomorphism by definition, which implies that the induced homomorphism $\bar{f}$ is an isomorphism.
\end{proof}

\section{Symmetric and Alternating Group}

\section{Classification of Groups of Small Order}

\section{Group Action on Sets}

\section{Sylow Theorems}

\section{Application of Sylow Theorems}

\section{Finite Simple Groups}

\end{document}
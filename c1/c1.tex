\documentclass{article}
\usepackage{../refalg}

\begin{document}
\Makepagesectionhead{MATH 594}{Group}{ARessegetes Stery}

\tableofcontents  
\newpage

\section{Group Preliminaries}

\begin{definition}[Group]
    A \textbf{group} is a set $G$ together with a binary operation $G \times G \to G$, often written $(a, b) \mapsto a \cdot b$ or simply $ab$, s.t. the following properties are satisfied:
    \begin{enumerate}
        \item \emph{Associativity:} $(ab)c = a(bc)$ for all $a, b, c \in G$.
        \item \emph{Existence of Identity}: There exists $e = e_G \in G$ s.t. $\forall a \in G, ae = a = ea$.
        \item \emph{Existence of Inverse}: For all $a \in G$. there exists $b \in G$ s.t. $ab = e = ba$.
    \end{enumerate}
    Furthermore, if the operation is commutative, i.e. for all $a, b \in G$, $ab = ba$, then the group is \textbf{commutative}, or \textbf{abelian}.
\end{definition}

\begin{remark}
    If the group $G$ is abelian, then the operation is often represented in additive notations (with operation denoted as ``$+$'', and inverse of $a \in G$ being $-a$).
\end{remark}

\begin{remark}
    One implicitly presented condition is that the operation of groups need to be closed within the set predefined. This is indicated by the signature of the operation, which should land in $G$. This often needs to be checked when the group structure is defined in some larger structure.
\end{remark}

\begin{remark}\label{rmk: group immediate results}
    From the definition of group there are some immediate facts/properties:
    \begin{enumerate}[label=\arabic*)]
        \item The identity in the group is unique. Suppose that there exist two identity elements $e$ and $e'$, then by rule 2, $e = ee' = e'$.
        \item For a given element in the group, the inverse of it is unique. Let $b$ and $b'$ both be the inverse of some $a \in G$. Then
        \[
            b = b(ab') = (ba)b' = b'    
        \]
        the uniqueness allows us to unambiguously denote the inverse of $a$ as $a^{-1}$. This also implies $(a^{-1})^{-1} = a$, as clearly by the previous process $a$ is the inverse of $a^{-1}$; and the inverse is unique.
        \item $(ab)^{-1} = b^{-1} a^{-1}$. By the uniqueness of the inverse element, it suffices to check that the claimed inverse satisfies rule 2. This is indeed the case as
        \[
            (ab)(b^{-1}a^{-1}) = (a(bb^{-1}))a^{-1} = (ae)a^{-1} = aa^{-1} = e
        \]
        and for multiplication in the other sequence the checking is similar. 
        \item For $a, b, c \in G$, then $ab = ac \implies b = c$; and $ba = ca \implies b = c$. This results directly from the fact that $a$ is invertible; and multiplying on the left/right, respectively, $a$, gives the desired result.
    \end{enumerate}
\end{remark}

\begin{remark}\label{rmk: Z to group element}
    The associativity of operation in the groups gives the unambiguity of writing successive multiplications. Rigorously, when written $x_1\ldots x_n$ for $n \geq 2$, it is defined inductively on $n$ via specifying the result to be $(x_1\ldots x_{n-1})x_n$. The convention is that for $n = 0$ this is simply the identity.

    In particular one can unambiguously write out the power of an element:
    \[
        a^n := \begin{cases}
            \underbrace{a\ldots a}_{n}  & n > 0 \\
            e                           & n = 0 \\
            \underbrace{a\ldots a}_{-n} & n < 0 \\
        \end{cases}
    \]
    This gives $a^m \cdot a^n = a^{m + n}$ for all $m, n \in \Z$. The cases where $m$ and $n$ are of the same sign are clear; and for those of opposite sign, applying the same elimination process as Remark \ref{rmk: group immediate results} 3) gives the desired result.

    If $G$ is abelian, in additive notation we often denote $n \cdot a := a^n$.
\end{remark}

\begin{definition}
    If $G$ and $H$ are groups, a \textbf{group homomorphism} $f: G \to H$ is a map s.t. $f(a\cdot b) = f(a) \cdot f(b)$ for all $a, b \in G$. 
\end{definition}

\begin{proposition}\label{prop: grp homo preserve identity and inverse}
    If $f: G \to H$ is a group homomorphism, then $f(e_G) = e_H$, and $f(a^{-1}) = (f(a))^{-1}$.
\end{proposition}

\begin{proof}
    By Remark \ref{rmk: group immediate results} 4) and the property of identity, we have
    \[
        f(e_G) \cdot e_H = f(e_G) = f(e_G \cdot e_G) = f(e_G) \cdot f(e_G) \implies e_H = f(e_G)
    \]
    For the second statement, use the above result:
    \[
        e_H = f(e_G) = f(a \cdot a^{-1}) = f(a) \cdot f(a^{-1})
    \]
    By the definition $(f(a))^{-1}$ is the inverse of $f(a)$. By the uniqueness of inverse this gives $f(a^{-1}) = f(a^{-1})$.
\end{proof}

\begin{remark}
    Given $f: G \to H$, $g: H \to K$ which are both $f$ and $g$ are group homomorphisms, then $f \circ g$ is also a group homomorphism. This results from the fact that
    \[
        f(g(a \cdot b)) = f(g(a) \cdot g(b)) = f(g(a)) \cdot f(g(b))
    \]
    The fact that morphism is closed w.r.t. composition implies that the groups form a category \underline{Grps}.
\end{remark}

\begin{definition}
    If $G$ and $H$ are groups, then $f: G \to H$ is a \textbf{group isomorphism} if it is a bijective group homomorphism.
\end{definition}

\begin{proposition}\label{prop: categorical def of group isomorphism}
    $f: G \to H$ being a group homomorphism is a group isomorphism if and only if there exists a group homomorphism $g: H \to G$ s.t. $g \circ f = \Id_G$, and $f \circ g = \Id_H$.
\end{proposition}

\begin{proof}
    It suffices to show implication in two directions:
    \begin{enumerate}
        \item[$\Rightarrow$:] Since $f$ is bijective, there must admit a (pointwise) inverse of $f$ s.t. $f^{-1} \circ f = \Id_{G}$, $f \circ f^{-1} = \Id_H$. Define $g = f^{-1}$. It suffices to check that $g$ is a group homomorphism. To prove this we need to verify that for all $u, v \in H$, $g(u \cdot v) = g(u) \cdot g(v)$. Since $f$ is bijective, $f$ is in particular injective, i.e. $a = b$ if and only if $f(a) = f(b)$ for all $a, b \in G$. Therefore to verify the equality above it suffices to verify the equality after applying $f$, i.e. $f \circ g(u \cdot v) = f \circ g(u) \cdot f \circ g(v)$. Then the equality holds as $f \circ g = \Id_H$. 
        \item[$\Leftarrow$:] Prove the contrapositive. If $f$ is not injective, then $g$ cannot be well-defined; and if $f$ is not surjective, then the domain of the composition $f \circ g$ is not the whole $H$.
    \end{enumerate}
\end{proof}

\begin{remark}
    Recall that under the context of categories, isomorphisms are defined as in Proposition \ref{prop: categorical def of group isomorphism}. The same proposition implies that group isomorphisms are isomorphisms in the categorical sense. 
\end{remark}

\begin{remark}
    If there exists an isomorphism $f: G \to H$ between groups $G$ and $H$, then $G$ and $H$ are considered as \textbf{isomorphic}, denoted $G \cong H$. This is an equivalence relation as compositions of isomorphisms are still isomorphisms.
\end{remark}

\begin{definition}
    Let $G$ be a group. Then a \textbf{subgroup} of $G$ is a subset $H \subseteq G$, which is in it self a group; and the inclusion map $i: H \hookrightarrow G$ is a group homomorphism. $H$ being the subgroup of $G$ is denoted as $H \leq G$.
\end{definition}

\begin{remark}
    The fact that the inclusion map is required to be a group homomorphism implies that the operation in $H$ is simply the restriction of the operation in $G$.
\end{remark}

\begin{proposition}\label{prop: subgroup test}
    Let $G$ be a group, and $H \subseteq G$ a subset. Then the followings are equivalent:
    \begin{enumerate}[label=\roman*)]
        \item $H$ is a subgroup of $G$.
        \item The following three conditions are satisfied:
        \begin{enumerate}[label=\arabic*)]
            \item For all $a, b \in H$, $a \cdot b \in H$.
            \item $e_G \in H$.
            \item (Under the same operation of $G$) $a^{-1} \in H$ for all $a \in H$. 
        \end{enumerate}
        \item $H$ is nonempty; and for all $x, y \in H$, $x \cdot y^{-1} \in H$.
    \end{enumerate}
    The third condition is often used to test whether $H \subseteq G$ gives a subgroup. 
\end{proposition}

\begin{proof}
    Verify the following implications:
    \begin{itemize}
        \item i) \implies ii). By the definition of subgroup, $H$ together with the same operation is a group, which by the definition of group is closed w.r.t. the group; and every element should admit an inverse. By the fact that $i$ is an inclusion, and by Proposition \ref{prop: grp homo preserve identity and inverse} $i(e_H) = e_G$ with $e_G = e_H$. 
        \item ii) \implies i). Check that $H$ is a group: associativity is given by the fact that the operation is identical to that in $G$. and $G$ is a group; existence of inverse and identity results directly from hypothesis 2) and 3); and the operation is defined as $H \times H \to H$ given by hypothesis 1).  
        \item ii) \implies iii). By 2) $H$ is nonempty. For all $x, y \in H$, by 3) $y^{-1} \in H$; and by 1) $x \cdot y^{-1} \in H$ given that both $x$ and $y^{-1}$ are in $H$. 
        \item iii) \implies ii). Since $H$ is nonempty, there exists $a \in H$. iii) implies that $a \cdot a^{-1} = e_G \in H$, giving 2). For all $a \in H$, let $x = e_G$ and $y = a$, which gives $a^{-1} \in H$, satisfying 3). For all $a, b \in H$, letting $x = a, y = b^{-1}$ gives $a \cdot b \in H$.  
    \end{itemize}
\end{proof}

\begin{proposition}
    Let $f: G \to H$ be a group homomorphism, then if $G' \leq G$, then $f(G') \leq H$.
\end{proposition}

\begin{proof}
    Apply the result of Proposition \ref{prop: subgroup test}. Since $G' \leq G$, $e_G \in G'$, and bby Proposition \ref{prop: grp homo preserve identity and inverse}, $f(e_G) = e_H$, giving that $f(G')$ is nonempty. For all $x, y \in f(G')$, let $u, v \in G'$ s.t. $x = f(u), y = f(v)$. Since $G'$ is a subgroup of $G$, $u\cdot v^{-1} \in G'$. By Proposition \ref{prop: grp homo preserve identity and inverse}, this implies $f(u) \cdot f(v^{-1}) = f(u) \cdot f(v^{-1}) \in f(G')$, which gives that $f(G') \leq H$.
\end{proof}

\begin{proposition}
    Let $f: G \to H$ be a group homomorphism. If $H' \leq H$, then $f^{-1} (H') \leq G$. In particular, $f^{-1}(e_H) = \ker f := \{u \in G \mid f(u) = e_H\}$ is a subgroup of $G$.
\end{proposition}

\begin{proof}
    Apply the same argument as in the above proposition. $H' \leq H \implies e_H \in H' \implies e_G \in f^{-1}(H')$, i.e. $f^{-1}(H')$ is nonempty. For all $u, v \in f^{-1}(H')$, $f(u\cdot v^{-1}) = f(u) f(v)^{-1} \in H'$ since $H' \leq H$, which implies that $u \cdot v^{-1} \in f^{-1}(H')$, i.e. $f^{-1}(H')$ is a group. 
\end{proof}

\begin{proposition}
    Let $f: G \to H$ be a group homomorphism. Then $f$ is injective if and only if $\ker f = \{e_G\}$.
\end{proposition}

\begin{proof}
    Proceed by showing implication in both directions:
    \begin{enumerate}
        \item[$\Rightarrow$:] Let $u \in \ker f$. Then $f(a) = f(a) \cdot e = f(a) \cdot f(u) = f(a \cdot u)$. But $f$ being injective implies that $a = a\cdot u$, i.e. $u = e$.
        \item[$\Leftarrow$:] For $u, v \in G$ s.t. $f(u) = f(v)$, we have $e = f(u) \cdot (f(v))^{-1} = f(u) \cdot f(v^{-1}) = f(u \cdot v^{-1}) \implies that u \cdot v^{-1} \in \ker f$. But since the only element in $\ker f$ is the identity, this gives $u \cdot v^{-1} = e \implies u = v$, i.e. $f$ is injective.
    \end{enumerate}
\end{proof}

\section{Group of Permutations}

\begin{definition}
    Given a set $\Omega$, the \textbf{permutation group} is defined to be $S_{\Omega} := \{f: \Omega \to \Omega \mid f \text{bijection}\}$. Since compositions of bijective maps are still bijective, defining the operation to be composition gives this a group structure.
\end{definition}

\begin{remark}
    Notice that the permutation group structure depends only on the cardinality of the group on which permutations are considered. Explicitly, for $\alpha: \Omega \to \Omega'$ a bijection, there exists an isomorphism between the corresponding groups of permutations: $\beta: S_{\Omega} \to S_{\Omega'}: f \mapsto \alpha \circ f \circ \alpha^{-1}$. This is indeed an isomorphism as this is first a group homomorphism since
    \[
        \beta(f \circ g) = \alpha \circ f \circ g \circ \alpha^{-1} = \alpha \circ f \circ \alpha^{-1} \alpha \circ g \circ \alpha^{-1} = \beta(f) \circ \beta(g)
    \]
    and this being an isomorphism follows from the fact that there exists an obvious inverse $\beta^{-1}: f \mapsto \alpha^{-1} \circ f \circ \alpha$. Therefore it suffices to denote such permutation group by the cardinality of $\Omega$: for $\Omega = \{1, \dots, n\}$ $S_{\Omega}$ is denoted as $S_n$.
\end{remark}

\begin{proposition}[Cayley]
    Every group can be embedded into some $S_{\Omega}$. Explicitly, for group $G$ the map $\alpha: G \to S_G$ s.t. $g \mapsto \alpha_g$ where $\alpha_g(h) = gh$ ($\alpha_g$ is the action of $G$ on $G$ defined by multiplication by $g$.) is an injective group homomorphism.
\end{proposition}

\begin{proof}
    It suffices to syntactically check that the following requirements are satisfied:
    \begin{itemize}
        \item \emph{$\alpha_g \in S_G$.} It suffices to check that indeed multiplication by an element in the group gives a bijection. This is clear as the action has an inverse, namely multiplying the inverse of that element. 
        \item \emph{$\alpha$ gives a group homomorphism.} By definition $\alpha_{gh} = \alpha_g \cdot \alpha_h$.
        \item \emph{$\alpha$ is injective.} It suffices to check that $\ker \alpha = e_G$. This is indeed the case, as for $g \in G$ s.t. $\alpha_g = \Id$, $\alpha_g(e_G) = g \cdot e_G = e_G \implies g = e_G$.
    \end{itemize}
\end{proof}

\clearpage
\section{Groups Generated by a Subset}

\begin{remark}
    If $(H_i)_{i \in I}$ is a family of subgroups of $G$, then $\bigcap_{i \in I} H_i$ is also a subgroup of $G$. This can be verified by taking an element in the intersection, and check each rule of group in each of the $H_i$s.
\end{remark}

\begin{definition}
    If $A \subseteq G$ is a subset of $G$, then the \textbf{subgroup generated by $A$} is defined as
    \[
        \inner{A} := \bigcap_{A \subseteq H \leq G} H
    \]
\end{definition}

\begin{remark}
    By definition $\inner{A}$ is well-defined, as in particular $A \subseteq G \leq G$. By the previous remark, $\inner{A}$ is a subgroup of $G$. It is also the smallest subgroup that contains $A$.
\end{remark}

\begin{proposition}\label{prop: explicit presentation of generated subgroup}
    Let $A \subseteq G$ be a subset of $G$, then $\inner{A} = \left\{ x_1\dots x_n \mid n \in \Z_{>0}; \forall i, x_i \in G \text{ or } x_i^{-1} \in G \right\}$. For $n = 0$, $x_1 \dots x_n = e$. 
\end{proposition}

\begin{proof}
    Proceed by double inclusion:
    \begin{enumerate}
        \item[$\subseteq$:] Proceed to show that RHS satisfies the definition of the $H$s above. For RHS consider $n = 1$, with $x_1 \in G$ which takes all elements in $G$. This gives $A \subseteq$ RHS. Further use Proposition \ref{prop: subgroup test}, which for any $x_1\dots x_m, y_1\dots y_n \in$ RHS, each summand of $x_1 \dots x_m (y_1 \dots y_n)^{-1} = x_1 \dots x_m y_n^{-1} \dots y_1^{-1}$ is either in $A$ or its inverse is in $A$ implying that RHS is a group. Definition above gives the subset relation.
        \item[$\supseteq$:] It suffices to verify that any element in the specified form is in $\inner{A}$. This is the case as for $x_1\dots x_n$ where for all $i$, either $x_i \in A$ or $-x_i \in A$, $x_i \in \inner{A}$ by definition, and multiplication of two elements in the group is still in the group by closure of the operation.  
    \end{enumerate}
\end{proof}

\begin{definition}
    The following defines some common terminology for characterization of a group:
    \begin{itemize}
        \item $G$ is \textbf{finitely generated} if there exists a finite set $A \subseteq G$ s.t. $G = \inner{A}$.
        \item $G$ is \textbf{finite} if it has finitely many elements. 
        \item The \textbf{order} of $G$, denoted $\abs{G}$, is the number of elements in $G$ if it is finite; or $\infty$ if $G$ is not finite (infinite).
        \item $G$ is \textbf{cyclic} if it attains a generating set with a single element $a$. In this case $G$ is denoted as $G = \inner{A}$.
        \item The \textbf{order} of $x \in G$, denoted $\abs{a}$ is the order of $\inner{a}$.  
    \end{itemize}
\end{definition}

\begin{remark}
    Cyclic groups are abelian. By the alternative definition provided in Proposition \ref{prop: explicit presentation of generated subgroup}, $\inner{a} = \{ a^m \mid m \in \Z \}$.
\end{remark}

\begin{proposition}\label{prop: characterization of cyclic group}
    If a group $G$ is cyclic, then $G \simeq \Z$ if $G$ is infinite, or $G \simeq \Z/n\Z$ for some $n \in \Z_{>0}$.  
\end{proposition}

\begin{proof}
    Choose $a \in G$ s.t. $G = \inner{a}$. Proceed via showing implication in both directions:
    \begin{enumerate}
        \item[$\Rightarrow$:] Consider $f: \Z \to G$ s.t. $f(1) = a$. This is a group homomorphism,  Then either
        \begin{itemize}
            \item \emph{$f$ is injective.} By definition of cyclic groups, for any $s \in G$ there exists $m \in G$ s.t. $s = a^m$. Then $f(m) = s$ according to the definition of $f$, giving that $f$ is surjective. Then this falls into the first case, giving $G \simeq \Z$.
            \item \emph{$f$ is not injective.} Then there are nonzero elements in $f$. Since $\ker f \subseteq \Z$, there exists a smallest positive element. Define the map $f_n: \Z/n\Z \to G$ s.t. $[1] \mapsto a$. Check the followings:
            \begin{itemize}
                \item \emph{$f_n$ is well-defined.} It suffices to check that if $[m_1] = [m_2]$, then $f([m_1]) = (f[m_2])$. This is indeed the case as
                \[
                    f([m_1]) = a^{m_1} \overset{!}{=} a^{m_1} \cdot a^{(m_2 - m_1)} = a^{m_2} \cdot a^{nk} = a^{m_2} \cdot (a^n)^k = a^{m_2} = f([m_2])
                \]
                for some $k \in \Z$, where $\overset{!}{=}$ holds since $[m_1] = [m_2]$ implies $n \mid (m_1 - m_2)$. This gives $a^{m_1 - m_2} = e$ since $a^n = e$. 
                \item \emph{$f_n$ is injective}. For $a \in \Z$ s.t. $f_n([a]) = 0$, $a = 0$ as otherwise this conflicts with the hypothesis that $n$ is the smallest of such integers. 
                \item \emph{$f_n$ is surjective.} Follows from the same argument in the case where $G$ is infinite.
            \end{itemize}
        \end{itemize}
        \item[$\Leftarrow$:] Since $\Z = \inner{1}$ and $\Z/n\Z = \inner{[1]}$, both of which are cyclic. 
    \end{enumerate}
\end{proof}

\section{The Dihedral Groups}

\begin{definition}
    Let $n \geq 3$, and $P_n \subset \R^2 \simeq \C$ be the regular $n$-gon s.t. its vertices are at the $n$-th roots of 1. Then the \textbf{dihedral group} $D_{2n}$ is the group of symmetry of $P_n$. Alternatively, one can write
    \[
        D_{2n} = \{ \varphi \in \GL_2(\R) \mid \varphi(P_n) = P_n \}
    \]
\end{definition}

\begin{remark}
    We have a injective map $\alpha: D_{2n} \to S_n$, where $\alpha(\varphi)$ is given by the restriction of $\varphi$ to the vertices of $P_n$. This map is injective as $\{v_1, \dots, v_n\}$ spans $\R^2$. Therefore, specifying how the vertices are transformed (permuted) fixes the whole linear transformation.
\end{remark}

\begin{remark}
    Notice the following relations: by definition of rotation $\sigma^n = e$; and $\sigma \tau \sigma = \tau$, which implies $\sigma^{n-1} \tau = \tau \sigma$. This enables changing the sequence of applying $\sigma$s and $\tau$s.
\end{remark}

\begin{proposition}
    For a fixed $n$, let $\sigma$ be the operation of counter-clockwise rotation by $\frac{2\pi}{n}$ on $P_n$; and $\tau_j$ be the operation of symmetry w.r.t. the symmetry axis passing through the vertex $j$ (which is a direction; invariant w.r.t. transformations on $P_n$). Then for every $\alpha \in D_{2n}$, it must be in the form of $\sigma^i$ or $\sigma^1 \cdot \tau_j$, for some $i, j \in \Z$.
\end{proposition}

\begin{proof}
    How the operations permute the vertices is characterized by
    \[
        \sigma: v_k \mapsto v_{k + 1} \qquad \tau: v_{j+k} \mapsto v_{j-k}
    \]
    Following the strategy of the previous remark, to fix the whole operation $\alpha$ it suffices to fix how vertices are transformed. Since elements of $D_{2n}$ are linear transformations, they map line segments to line segments, and therefore adjacent vertices to adjacent vertices. Then for $v_1 \mapsto v_{i+1}$, either $v_2 \mapsto v_{i+2}$, then $\alpha = \sigma^i$; or $v_2 \mapsto v_{i}$, then $\alpha = \sigma^i \tau_j$. The indices are considered modulo $n$ and then plus 1. 
\end{proof}

\end{document}
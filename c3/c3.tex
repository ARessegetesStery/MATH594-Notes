\documentclass{article}
\usepackage{../refalg}

\begin{document}
\Makepagesectionhead{MATH 594}{Finite Fields and Galois Theory}{ARessegetes Stery}

\tableofcontents
\newpage

\section{Review of Ring Theory}

\textstart
The Galois Theory originates from the question: Given a polynomial with coefficients in a field (e.g. $\Q$), we want to understand the property of its solutions, and the algebraic structure in which the root lies. This section lays some fundamental notions and results for introducing the whole theory.

Setup the convention. All rings have unit element 1; and all ring homomorphisms map 1 to 1. Inclusion maps, in particular, implies that all subrings of a certain ring must include 1.

\begin{remark}
    An immediate result is that finite domains that are not the zero ring are fields.

    Recall that a ring $R$ is a domain if and only if zero does not have non-trivial divisors. That is, for all $x, y \in R \smallsetminus \{0\}$, $xy = 0$. For $a \in R$, $a \neq 0$, consider the map 
    \[
        \varphi: R \to R \qquad x \mapsto ax
    \]
    Since $R$ is a domain $\varphi$ maps nonzero elements to nonzero elements, which implies that there exists some $x_a$ s.t. $a x_a = 1$. This gives an inverse of $a$.
\end{remark}

\begin{proposition}\label{prop: ring homs from a field are injective}
    If $f: K \to R$ is a ring homomorphism, and $K$ is a field. Then $R \neq \{0\}$ implies that $f$ is injective.
\end{proposition}

\begin{proof}
    Recall that the kernel of a particular ring homomorphism is an ideal. Denote $I = \ker (f) \subseteq K$. Suppose that $a \in I$ s.t. $a$ is nonzero. Then $a^{-1} \in I$ which gives $I = (1) = K \implies f(I) = 0$. But as we requre $f(1) = 1_R \implies 1_R = 0_R$, i.e. $R = \{0\}$ which is a contradiction.
\end{proof}

\begin{corollary}
    In particular, ring homomorphisms between fields (field extensions) $K \to L$ are injective. Not all fields have extensions (ring homomorphisms) between them.
\end{corollary}

\textstart
A class of extensions of which we are particularly interested in, is the extension which gives polynomial a root. $f \in K[x]$ may not have a root; and by extending it $K[x] \ext L[x]$ we may consider roots of $f \in L[x]$ which may have a root.

\begin{notation}
    A field extension $k \ext K$ is also denoted as $K/k$. These two notations will be used interchangeably.
\end{notation}

\begin{proposition}
    Let $k$ be a field, and $R = k[X]$. Let $f \in R \smallsetminus \{0\}$. Then the followings are equivalent:
    \begin{enumerate}[label=\arabic*)]
        \item $f$ is irreducible.
        \item $(f)$ is a prime ideal.
        \item $(f)$ is a maximal ideal.
    \end{enumerate}
\end{proposition}

\begin{proof}
    Prove the implications cyclically:
    \begin{itemize}
        \item \emph{3) \implies 2).} It is a general fact that maximal ideals are prime. Prove the contrapositive: suppose that an ideal $(f) \subset R$ is not prime, then there exists $a, b \in R$ s.t. $ab \in (f)$ and neither $a$ and $b$ are in $(f)$. Then $(f) \subset (a) \subset R$ which implies that $(f)$ is not maximal.
        \item \emph{2) \implies 1).} $(f)$ being a prime ideal in $R$ implies that in particular $(f) \neq R$, i.e. $f$ is not invertible. Suppose that there exists $g, h$ not invertible s.t. $f = gh$. Without loss of generality, assume that $g \in (f)$. Then there exists some $u \in R$ s.t. $g = fu$. Multiply on the right by $h$ gives $f = gh = fuh$ which implies that $h$ is invertible, giving a contradiction.
        \item \emph{3) \implies 1).} $f$ being irreducible implies that $f$ is not invertible, i.e. $(f) \neq R$. Suppose that there exists some maximal ideal $J$ s.t. $(f) \subset J \subset R$. Then since maximal ideals are in particular prime, $J = (g)$ for some $g \in R$. But then this implies that $f = gu$ giving that $u$ is invertible, and therefore $(f) = J$, which is a contradiction.
    \end{itemize}
\end{proof}

\textstart
Recall that two elements $f, g$ are \underline{relative prime} if for all $p \in R$ s.t. $p \mid f$, $p \mid g$, $p$ is invertible. Then we have the following result similar to the case for integer divisibility:

\begin{proposition}\label{prop: divisibility for relative prime polynomials}
    Let $k$ be a field, and $R = k[x]$. For $f, g, h \in R$ s.t. $f \mid gh$, if $f$ and $g$ are relative prime, then $f \mid h$.
\end{proposition}

\begin{proof}
    Since $k$ is a field, $k[x]$ is a PID (as every element in the coefficient is invertible, for $a, b$ relative prime $(a, b) = (1) = k[x]$). Consider $I = (f, g) \subseteq R$. Since $I$ is principal, $I = (p)$ for some $p \in k[x]$. Therefore $p \mid f$, $p \mid g$, which implies that $p$ is invertible. Then $I = R$. This gives that there exists $A, B \in R$ s.t. $Af + Bg = 1$. Multiplying $h$ on the right gives $Afh + Bgh = h$. Since $f$ divides LHS, $f \mid h$.
\end{proof}

\begin{proposition}
    Let $k$ be a field, and $R = k[x]$. If $f \in R \smallsetminus \{0\}$, and denoting $d = \deg f$, then there exists field extensions $k \ext L$ and $a_1, \dots, a_d \in L$, $c \in k$ s.t. $f = c(x - a_1)\cdots (x - a_d)$ in $L[x]$.
\end{proposition}

\begin{proof}
    The key step is to show that if $f$ is irreducible in $R$, then there exists a field extension $k \ext k'$ s.t. $f$ has a root in $k'$.

    Consider $k' = k[x]/(f)$. Since $f$ is irreducible, $(f)$ is a maximal ideal in $R$, and therefore $k'$ is a field. Since elements in $k$ are of degree 0 in $k[x]$. Considering $k \too k[x] \too k' := k[x]/(f)$ gives the injective ring homomorphism (field extension). Let $a = \overline{x} \in k'$. Then $f(a) = \overline{f(x)} = 0$, which implies that $a$ is a root of $f$.

    Proceed the rest of the proof by induction: 
    \begin{itemize}
        \item \emph{$d = 0$.} This is the trivial case.
        \item \emph{$d = 1$.} Then $f = c(x - a)$ for some $c, a \in k$.
        \item \emph{$d \geq 2$.} Apply the above steps iteratively. Then there exists some $a \in k'$ s.t. $(x - a) \mid f$, i.e. in $k'[x]$ we have the decomposition $f = (x - a) g$ for some $g \in k'[x]$ with $\deg g = \deg f - 1$. Applying the inductive hypothesis (the results holds in lower degrees) gives the full decomposition in $L := k'$. 
    \end{itemize}
\end{proof}

\begin{notation}\label{not: extending field by algebra}
    Denote the image of the map $(k[y] \to k, y \mapsto a)$ by $k[a]$. This is the smallest $k$-algebra containing $a$.
\end{notation}

\begin{proposition}\label{prop: existence of field extending root of polynomial}
    Let $k$ be a field, and $R = k[x]$. Let $f \in R \smallsetminus \{0\}$ be irreducible. Suppose that we have the field extension $k \ext K$, and $a \in K$ is a root of $f$. Then $k[a] \simeq k[x]/(f)$. In particular, $k[a]$ is a field.
\end{proposition}

\begin{proof}
    Consider the ring homomorphism $\varphi: k[y] \to K$ s.t. $\varphi(y) = a$. Then by the First Isomorphism Theorem, we have $k[a] = \im \varphi \simeq k[x]/\ker \varphi \simeq k[x]/(f)$ since $f(a) = 0$. 
\end{proof}

\textstart
Notice that every field extension $k \ext K$ is a $k$-algebra morphism (ring homomorphisms that are $k$-linear). Since $k$ is a field, this gives $K$ a $k$-vector space structure.

\begin{definition}[Degree]
    The \textbf{degree} of the field extension $k \ext K$, denoted $[K : k]$, is $\dim_k K \in \Z_{\geq 0}$ or infinite. 
\end{definition}
\nogap
\begin{definition}[Finite]
    A field extension is \textbf{finite} if the degree of it is finite. 
\end{definition}

\begin{remark}\label{rmk: degree of minimal polynomial gives degree of extension}
    If $f \in k[x]$ is irreducible, and $K = k[x]/(f)$, then $[K : k] = \deg f$. More generally, if $g \in k[x]$ is a nonzero polynomial, then $\dim_k(k[x]/(g)) = \deg g$.

    This can be seen via applying the division algorithm (since $K[x]$ is an Euclidean Domain. This can be seen via computing the division). Then for all $P \in k[x]$, there exists unique $Q, R \in k[x]$ s.t. $P = gQ + R$, with $\deg R < \deg g$. Then since $\overline{P} = \overline{R}$ in $k[x]/(g)$, $\{ \bar{1}, \bar{x}, \dots, \overline{x^{\deg g - 1}} \}$ gives a basis of $k[x]/(g)$ over $k$.
\end{remark}

\section{Multiplicity of Root}

\textstart
This section provides tools for describing the zeros of a polynomial, and how they in general can look like. The proposition below says that any polynomial can be factored into two parts, with the first part having roots in the field; and the second part requires extension of the field to decompose completely. 

\begin{definition}[Multiplicity]
    Let $f \in k[x]$ be a nonzero polynomial for $k$ a field, and $a \in R$ a root of $f$, Then $a$ has \textbf{multiplicity} $m$ if $(x - a)^m \mid f$, but $(x - a)^{m + 1} \nmid f$.
\end{definition}

\begin{proposition}
    If $f \in R \smallsetminus \{0\}$, and $a_1, \dots, a_r \in k$ are pairwise distinct roots of $f$ s.t. $a_i$ has multiplicity $m_i$. Then we have the decomposition of $f$: 
    \[
        f = \prod_{i = 1}^r (x - a_i)^{m_i} g, \qquad g \in R, \text{$g(a_i) \neq 0$ for all $i$}
    \]
    In particular, $\sum_{i} m_i \leq \deg f$.
\end{proposition}

\begin{proof}
    Apply induction on $r$: 
    \begin{itemize}
        \item \emph{Base case.} Then $m_1$ is the maximal integer satisfying the condition that $(x - a_1)^{m_1} \mid f$. Then define $g$ be such that $f = (x - a_1)^{m_1} g$. 
        \item \emph{Inductive step.} For $r \geq 2$,  denote $f_1$ be the polynomial s.t. $f = (x - a_1)^{m_1} f_1$. Notice that for all $i$ s.t. $2 \leq i \leq r$, we have $(x - a_i)^{m_i} \mid f$. Then since $(x - a_i)$ and $(x - a_1)$ are relative prime (they are both irreducible) by Proposition \ref{prop: divisibility for relative prime polynomials} we have $(x - a_i)^{m_i} \mid f_1$. Then applying inductive hypothesis gives the desired decomposition of $f$. 
    \end{itemize}
\end{proof}

\section{Characteristic of a Field}

\textstart
Recall that in the first section we mentioned that there does not necessarily exist ring homomorphisms between arbitrary fields. This, as we will see in the following, implies some constraints on the structure that a field can have.

Let $S$ be an integral domain. Let $\varphi: \Z \to S$ s.t. $n \mapsto n \cdot 1_S$. This is the unique ring homomorphism between $\Z$ and $S$ due to the constraint the $1$ should be mapped to 1. Since $S$ is a domain, and $\Z$ is a PID, $\ker \varphi = (d)$ for $d$ prime or zero. Then either 
\begin{enumerate}[label=\arabic*)]
    \item $\ker \varphi = \{0\}$; or
    \item $\ker \varphi = p\Z$ for some $p$ prime.
\end{enumerate}

In case 1), if we suppose further that $S = k$ which is a field, then for all $n \in \Z$ $\varphi(n)$ is invertible. By the universal property of the quotient ring, this induces a ring homomorphism (which is also a field extension) $\Frac(\Z) = \Q \ext S$. 

In case 2), we have an injective ring homomorphism $\Z/p\Z \ext S$ for some $p$ prime. Defining $\F_p = \Z/p\Z$, $S$ becomes an $\F_p$-algebra by the $\varphi$ above. 

\begin{definition}[Characteristic]
    For a field $k$, the \textbf{characteristic} of $k$ is
    \[
        \fchar(k) = 
        \begin{cases}
            p, & \text{if $\F_p \ext k$ (case 1)} \\
            0, & \text{if $\Q \ext k$ (case 2)}
        \end{cases}
    \]
\end{definition}

\begin{remark}
    If $S$ is an $\F_p$-algebra (case 2), the map $F: S \to S$, $u \mapsto u^p$ is the \underline{Frobenius homomorphism}. Check that this is indeed a ring homomorphism:
    \begin{itemize}
        \item $F(uv) = F(u) F(v)$. Clear as field is commutative: $(uv)^p = u^p v^p$.
        \item $F(u + v) = F(u) + F(v)$. Compute: 
        \[
            (u + v)^p = u^p + v^p + \underbrace{\sum_{i = 1}^{p - 1} \binom{p}{i} u^{p-i} v^i}_{\text{divisible by $p$}}
        \]
        where the last term vanishes, as $\F_p \ext S$ should map $0$ to $0$; and $\bar{p} = \bar{0} \in \F_p$.
    \end{itemize}
\end{remark}

\section{Algebraic Extensions}

\textstart
The field extensions originating solely from ``including the roots of polynomials'' are the nice ones and deserve a better name. The discussions formalizes the concept of ``algebraic closure'' in elementary discussions of polynomials.

\begin{proposition}\label{prop: degree of composition of field extensions}
    If $k \ext K \ext L$ is a field extension, then $[L:k] = [L:K][K:k]$.
\end{proposition}

\begin{proof}
    First consider the cases where one of the degrees is infinite: 
    \begin{itemize}
        \item If $[K : k]$ is infinite, then $[L : k]$ is infinite as $K \subseteq L$ is a $K$-vector subspace of $L$. 
        \item If $[L : K]$ is infinite, then there exists an infinite set of elements which are linearly independent over $K$, which are also linearly independent over $k$ since $k \subseteq K$.
    \end{itemize}
    Now consider the case where both $[L:K]$ and $[K:k]$ are finite. Denote $m = [L:K]$ and $n = [K:k]$. Denote $\{a_1, \dots, a_m\}$ be a basis of $L$ over $K$, and $\{b_1, \dots, b_n\}$ be a basis of $K$ over $k$. Notice that $\{a_i b_j \mid 1 \leq i \leq m, 1 \leq j \leq n\}$ gives a basis for $L$ over $k$, as for all $u \in L$, there exists $\lambda_i \in K$, and thus $\mu_{ij} \in k$ s.t.
    \[
        u = \sum_{i = 1}^n \lambda_i b_i = \sum_{i, j} \mu_{ij} a_j b_i, \qquad \text{for $\lambda_i = \sum_{j} \mu_{ij} a_j \mu_{ij}$}
    \]
    which is a decomposition. They are further linearly independent, as for $u = 0$, since $b_i$s give a basis, $\lambda_i = 0$ for all $i$, and therefore $\mu_{ij} = 0$ for all $i$ and $j$.
\end{proof}

\begin{notation}\label{not: extending field by field}
    Let $k \ext K$ be a field extension, and $A \subseteq K$ a subset. Then we denote
    \[
        k(A) := \bigcap_{A \subseteq k'} \left\{ k' \mid k \ext k' \ext K \text{ extension} \right\}
    \]
    which is the smallest field sub-extension of $k$ inside $K$ containing $A$. 
\end{notation}

\begin{remark}
    It is worth mentioning that this is different from $k[A]$ which is the smallest \emph{$k$-subalgebra} containing $A$:
    \begin{itemize}
        \item They are related via $k(A) = \Frac(k[A])$. They are equal in some ``nice'' extensions (see Remark \ref{rmk: algebraic implies k[a] = k(a)} below).
        
        By definition we have $k[A] \subseteq k(A)$, as $k[A]$ is only required to be a $k$-algebra instead of a field extension of $k$ (as field extending $k$ can be seen as $k$-vector spaces, which are in particular $k$-algebras). By the universal property of fraction fields, we have $\Frac(k[A]) \subseteq k(A)$, as ring homomorphisms between fields are injective, and by definition for all $f \in k[A]$, $f$ has an inverse in $k(A)$. Further since $A \subseteq \Frac(k[A])$ (also by definition, we have $k(A) = \Frac{k[A]}$).
        \item Considering multiple elements, we can extend the \hyperref[not: extending field by algebra]{previous notation}, by considering for $A = \{a_1, \dots, a_n\}$, then $k[A] = \im \varphi$ for
        \[
            \varphi: k[x_1, \dots, x_n] \to K, \qquad x_i \mapsto a_i
        \]
    \end{itemize}
\end{remark}

\begin{definition}[Finite Generated]
    A field extension $K/k$ is \textbf{finitely generated} if there exists $a_1, \dots, a_n \in K$ s.t. \newline $k(a_1, \dots, a_n) = K$.
\end{definition}

\begin{remark}
    If a field extension $K/k$ is finite, then it is also finitely generated, as $K/k$ being finite implies that there exists some finite basis of $K$ over $k$; and picking one gives the elements that ``finitely generates'' $K$. However, the converse is not true: consider $k \ext k(x) = \Frac(k[x])$ is finitely generated (by $x$) but is not finite (we have the infinite set $\{ x^i \mid i \in \Z \}$ whose elements are linearly independent over $k$)
\end{remark}

\begin{definition}[Algebraic; Transcendental]
    Let $k \ext K$ be a field extension. An element $a \in K$ is \textbf{algebraic over $k$} if there exists $f \in k[x] \smallsetminus \{0\}$ s.t. $f(a) = 0$ in $K[x]$. Otherwise $a$ is \textbf{transcendental}. An extension $K/k$ is \textbf{algebraic} if for all $a \in K$, it is algebraic over $k$.
\end{definition}

\begin{remark}
    Consider the field extensions $k \ext K \ext L$. Then if $a \in L$ is algebraic over $k$, then $a$ is also algebraic over $K$, as $a$ algebraic over $k$ implies that there exists $f \in k[x]$ s.t. $f(a) = 0$; and by definition we also have $f \in K[x]$.
\end{remark}

\begin{remark}\label{rmk: kernel of evaluation is prime}
    Given a field extension $k \ext K$, and $a \in K$. Then $a$ is algebraic if and only if the $\varphi: k[x] \to k, x \mapsto a$ has a non-trivial kernel. This is the direct translation of having a polynomial $f$ with $a$ as its root. Then $\ker \varphi$ is a prime ideal.
        
    To prove this, it suffices to show that $k[x]/(\ker \varphi)$ is a domain. This is indeed the case, as $k[x]$ is a domain: for all $g, h \in k[x]$, $g(a) \neq 0$ and $h(a) \neq 0$ implies that $gh(a) \neq 0$, i.e. $gh \notin \ker \varphi$ ($gh \neq \bar{0}$ in $k[x]/(\ker \varphi)$). Therefore, there exists some $f$ s.t. $\ker \varphi = (f)$.
\end{remark}

\begin{definition}[Minimal Polynomial]
    $f \in k[x]$ is the \textbf{minimal polynomial} of $a \in K$ if for $\varphi: k[x] \to K$, $x \mapsto a$, $\ker \varphi = (f)$. This is well-defined by the above Remark (Remark \ref{rmk: kernel of evaluation is prime}).
\end{definition}

\begin{remark}\label{rmk: algebraic implies k[a] = k(a)}
    For $a$ being algebraic, $f$ is a maximal ideal (by the fact that $k[x]/(\ker \varphi)$ is a domain). Then $k[a] = k[x]/(\ker \varphi)$ is a field. This gives $k[a] = k(a)$.
\end{remark}

\begin{remark}
    Given field extension $K/k$, if $a \in K$ is transcendental over $k$, then $k[a] \simeq k[x]$, as since there is no polynomial with root $a$ implies that $k[a]$ can be seen as injecting a formal variable to the field $k[x]$. This further implies $k(a) \simeq k(x)$ (as $k(a) \simeq \Frac(k[a])$; and same for $x$).
\end{remark}

\begin{example}
    Suppose that $d \in \Z$ is not a square, and let $a = \sqrt{d}$. Consider the field extension $\Q \to \C$.
    
    Since $a$ is not in $\Q$, $a$ cannot be a root of degree 1 polynomials, i.e. the minimal polynomial of $a$ must be of degree at least 2; and we have $a$ as a root of $x^2 - d = 0$, the minimal polynomial of $a$ over $k$ is $x^2 - d = 0$, which also implies that $a$ is algebraic over $\Q$.

    Therefore, $\Q(\sqrt{d})$ can be seen as a $\Q$-vector space, which has a basis $\{1, \sqrt{Q}\}$, i.e. $\Q(\sqrt{d}) = \{ a + b \sqrt{d} \mid a, b \in \Q \}$. In particular we have $\dim_{\Q} \Q(\sqrt{d}) = 2$ as $\deg f = 2$, resulting from Remark \ref{rmk: degree of minimal polynomial gives degree of extension}.
\end{example}

\begin{proposition}\label{prop: finite extensions are algebraic}
    Every finite extension $K/k$ is algebraic. 
\end{proposition}

\begin{proof}
    Denote $[K : k] = n$ which is finite. Regarding $K$ as a $k$-vector space, for all $a \in K$ the set of elements $\{ 1, a, \dots, a^n \}$ are linearly dependent (as there are $(n+1)$ of them). That is, there exists $c_0, \dots, c_n \in k$ s.t. $c_0 + c_1 a + \cdots + c_n a^n = 0$; and $f = c_0 + c_1 x + \cdots + c_n x^n \in k[x] \smallsetminus \{0\}$. This gives a polynomial $f \in k[x]$ s.t. $f(a) = 0$, i.e. $a$ is algebraic. Since $a \in K$ can be taken arbitrarily, we have $K/k$ being algebraic. 
\end{proof}

\begin{proposition}\label{prop: finitely generated algebraic extensions are finite}
    Let $k \ext K$ be a field extension, and $a_1, \dots, a_n \in K$ are algebraic over $k$. Then $k[a_1, \dots, a_n] = k(a_1, \dots, a_n)$; and $k(a_1, \dots, a_n)$ is a finite extension over $k$. 
\end{proposition}

\begin{proof}
    Remark \ref{rmk: algebraic implies k[a] = k(a)} gives the case for $n = 1$, $k[a] = k(a)$, and Remark \ref{rmk: degree of minimal polynomial gives degree of extension} (using minimal polynomial) gives that the extension is finite. 

    For $n \geq 2$, repeat the argument with the induction hypothesis. First by definition we have $k[a_1, \dots, a_n] = (k[a_1, \dots, a_{n-1}]) [a_n]$; and the same holds for the field extension version (replace brackets with parentheses). Further finite extensions are transitive, by Proposition \ref{prop: degree of composition of field extensions}. Suppose now that $k' = k[a_1, \dots, a_{n-1}] = k(a_1, \dots, a_{n-1})$ is a finite extension over $k$. Then $k[a_1, \dots, a_n] = k'[a_n] = k'(a_n) = k(a_1, \dots, a_n)$ is finite over $k'$; and by hypothesis we know $k'[a_n]$ is finite over $k$ since $k'$ is. 
\end{proof}

\begin{corollary}
    Finitely generated algebraic field extensions are finite.
\end{corollary}

\begin{proposition}\label{prop: transitivity of algebraic extensions}
    If $k \ext K \ext L$ are both algebraic field extensions, then so is $k \ext L$.
\end{proposition}

\begin{proof}
    Use the above two propositions. Let $a \in L$. Since $L/K$ is algebraic, there exists $f \in K[x] \smallsetminus \{0\}$ s.t. $f(a) = 0$. Let $f = c_0 + c_1 x + \cdots + c_n x^n$, and $k' = k(c_0, \dots, c_n)$. Since all $c_0, \dots, c_n \in K$ are algebraic over $k$, Proposition \ref{prop: finitely generated algebraic extensions are finite} implies that $k'$ is a finite extension over $k$. Further since $a$ is algebraic over $k'$, the extension $k' \ext k'(a)$ is also finite. Proposition \ref{prop: finite extensions are algebraic} implies that the extension $k \ext k'(a)$ is algebraic. That is, $a$ is algebraic over $k$.

    Since this holds for all $a \in L$, the extension $L/k$ is algebraic.
\end{proof}

\begin{remark}\label{rmk: subextension of algebraic extension is algebraic}
    Notice that the converse of the statement is also true. If $k \ext L$ is algebraic, then for every element $a \in L$ there exists some $f_a \in k[x]$ s.t. $f_a(a) = 0$. But given extension $k \ext K \ext L$, $k \subseteq K$, which implies $f_a$ is also in $K[x]$; and therefore $L/K$ is algebraic
\end{remark}

\begin{proposition}\label{prop: algebraic closure is a field}
    If $k \ext K$ is a field extension, then $k' = \{a \in K \mid \text{$a$ algebraic over $k$}\}$ is a subfield of $K$ containing $k$,
\end{proposition}

\begin{proof}
    To prove this we need to check:
    \begin{itemize}
        \item $k \subseteq k'$. This is clear from the construction of the field.
        \item $k'$ is closed under additive and multiplicative inverse. 
        
        $a \in k'$ implies $-a \in k'$ as $-a$ is the root of a polynomial via considering the minimal polynomial and inverting the corresponding coefficients. $a \in k' \smallsetminus \{0\}$ implies $a^{-1} \in k' \smallsetminus \{0\}$ since $k[a] = k(a)$ as $k(a)$ is finitely generated, and therefore finite and algebraic. $k(a)$ is a field, implying that $a^{-1} \in k(a)$.
        \item $k'$ is closed under addition and multiplication. That is, for all $a, b \in k'$, $a + b \in k'$ and $ab \in k'$.

        Consider the field $k(a, b)$. Since both $a$ and $b$ are algebraic over $k$, $k(a, b)/k$ is finite by Proposition \ref{prop: finitely generated algebraic extensions are finite}; and by Proposition \ref{prop: finite extensions are algebraic} the extension is algebraic. By definition $k(a, b) \subseteq k'$; and is the smallest field containing both $a$ and $b$; and therefore both $a + b$ and $ab$ are in $k'$.
    \end{itemize}
\end{proof}

\begin{definition}[Algebraic Closure]\label{def: algebraic closure with ambient field}
    For a field $k$ with field extension $k \ext K$, the field $k' = \{a \in K \mid \text{$a$ algebraic over $k$}\}$ is the \textbf{algebraic closure} of $k$ in $K$. By the above Proposition \ref{prop: algebraic closure is a field}, this is indeed a field.
\end{definition}
\nogap
\begin{definition}[Algebraically Closed]
    A field $k$ is \textbf{algebraically closed} if every nonzero polynomial over $k$ has a \mbox{root in $k$.}
\end{definition}
% using \mbox to suppress newline (otherwise there will be a lagging $k$ in the new line)

\begin{remark}
    By induction, if $k$ is algebraically closed and $f$ is a nonzero polynomial in $k[x]$ with degree $n$< then there exists $a_1, \dots, a_n \in k$, $c \in k^{\ast}$ s.t. $f = c(x - a_1) \cdots (x - a_k)$. That is, every irreducible polynomial in an algebraically closed field has degree 1.
\end{remark}

\begin{notation}
    For a field $k$, the set of invertible (nonzero) elements in it are often denoted as $k^{\ast}$ or $k^{\times}$.
\end{notation}

\begin{proposition}
    A field $k$ is algebraically closed if and only if for all field extensions $k \ext K$, for all $a \in K \smallsetminus k$, $a$ is transcendental over $k$.
\end{proposition}

\begin{proof}
    Prove implication in two directions:
    \begin{itemize}
        \item[$\Rightarrow$:] Suppose that $k$ is algebraically closed, and $a \in K \smallsetminus k$ algebraic over $k$. Then by definition there exists an irreducible polynomial $f \in k[x]$ s.t. $f(a) = 0$. Since $k$ is algebraically closed, the irreducible polynomials are of degree 1, i.e. $\deg f = 1$. Then $a \in k$, which is a contradiction.
        \item[$\Leftarrow$:] Suppose that for all $a \in K \smallsetminus k$, $a$ is transcendental. Proceed to prove that $k$ is algebraically closed by showing that every irreducible polynomial has a root. 
        
        Consider $f \in k[x] \smallsetminus \{0\}$. Consider $K = k[x]/(f)$. Then since $f(\bar{x}) = 0$, $\bar{x}$ is algebraic over $k$. But since all elements in $K \smallsetminus k$ are transcendental, $\bar{x} \in k$. Then in $K$ we have $\bar{x} - a = 0$, for some $a \in k$. Then $f(x) = x - a$ in $K[x]$ which has a root $a$; and as this holds for all $f$, $k$ is algebraically closed.
    \end{itemize}
\end{proof}

\textstart
In summary, a field $k$ being algebraically closed is equivalent to the following conditions:
\begin{enumerate}[label=\arabic*)]
    \item For all field extensions $k \ext K$, either $k \simeq K$, or the extension is not algebraic. 
    \item For all $f \in k[x] \smallsetminus \{0\}$, $f$ factors as a product of polynomials of degree 1.
    \item Every irreducible $f \in k[x] \smallsetminus \{0\}$ has a root in $k$. 
\end{enumerate}

Recall that we have the \hyperref[def: algebraic closure with ambient field]{algebraic closure in a specific field}. The following theorem seeks to construct such closure without any ambient structure:

\begin{theorem}\label{thm: algebraic closure exists}
    Given any field $k$, there exists an algebraic extension $k \ext \bar{k}$ s.t. $\bar{k}$ is algebraically closed. Such an extension is an \underline{algebraic closure} of $k$.
\end{theorem}

\begin{parenthesis}[Direct Limit (of rings)]\label{pth: direct limit}
    To provide the construction of an algebraic closure, the main step is to iteratively include the roots of the some irreducible polynomials; and we need to ensure that this process terminates. That is, there exists a field containing all the intermediate fields on which we conducted the extension. This parenthesis formalizes this idea.

    \begin{definition}[Directed Set]
        $(I, \leq)$ is a \textbf{directed set} if for all $i, j \in I$ there exists $k \in I$ s.t. $i \leq k$ and $j \leq k$.
    \end{definition}

    \begin{definition}[Direct System]
        Given a directed set $I$, a \textbf{direct system} $(R_i)_{i \in I}$ is a family of rings $R_i$ satisfying:
        \begin{itemize}
            \item For all $i \leq j$ in $I$, there exists a ring homomorphism $\varphi_{ij}: R_i \to R_j$; and $\varphi_{ii} = \Id_{R_i}$.
            \item For all $i \leq j \leq k$ in $I$, the ring homomorphisms above satisfy $\varphi_{ik} = \varphi_{jk} \circ \varphi_{ij}$.
        \end{itemize}
    \end{definition}

    \begin{definition}[Direct Limit]
        Given a direct system $(R_i)_{i \in I}$, the \textbf{direct limit} of the system, denoted $R = \varinjlim R_i$, together with a family of ring homomorphisms $f_i: R_i \to R$ s.t.
        \begin{itemize}
            \item $f_j \circ \varphi_{ij} = f_i$.
            \item The pair $(R, (f_i)_{i \in I})$ is universal with this property, i.e. every ring homomorphism from $R_i$ for all $i$ factors uniquely through $R$. That is, for any ring $T$ and family of ring homomorphisms $g_i: R_i \to T$ s.t. $g_j \circ \varphi_{ij} = g_i$ (the $g_i$s are compatible w.r.t. the system), there exists a unique ring homomorphism $g: R \to T$ s.t. $g \circ f_i = g_i$ for all $i$.
        \end{itemize}
    \end{definition}

    The direct limit exists, by considering the class of elements that are closed along the morphisms of $R_i$s: define $R := \left(\bigsqcup_{i \in I} R_i\right)/\sim$, where the equivalence relation $\sim$ is given by $R_i \ni x_i \sim \varphi_{ij}(x_i)$. The operations are given by for $a_i \in R_i$ and $b_j \in R_j$, finding $k$ s.t. $i \leq k$ and $j \leq k$ (which exists as $I$ is a directed set), and define:
    \[
        a_i + b_j := \varphi_{ik}(a_i) + \varphi_{jk}(b_j) \in R_k, \qquad
        a_i b_j := \varphi_{ik}(a_i) \cdot \varphi_{jk}(b_j) \in R_k
    \]
    Existence of inverse, and distributivity are guaranteed by the ring structure of $R_j$. Further see that this is well-defined, as for $k' \neq k$ that we have picked s.t. $i \leq k'$ and $j \leq k'$, since the set is directed there exists $\ell$ s.t. $k \leq \ell$ and $k' \leq \ell$. Now use the equivalence relation with $\varphi_{k\ell}$ and $\varphi_{k'\ell}$; and the fact that $\sim$ is an equivalence relation and is therefore transitive. 

    Further the direct limit of a direct system is unique up to isomorphism. Suppose we have two direct limits $R$ and $R'$, there exists $g: R \to R'$ and $g': R' \to R$, by the universal property of the direct limit s.t. $g \circ f_i = f_i'$, $g' \circ f_i' = f_i$. This implies that $g \circ g' = \Id_{R'}$, and $g' \circ g = \Id_{R}$ by the symmetric argument. This gives the isomorphism.

    The last thing we need to notice is that if all $R_i$s are fields, then $R$ is a field, as in the definition of the operation having multiplicative inverse in $R_k$ induces the multiplicative inverse in $R$.
\end{parenthesis}

Now we can use the above tools to describe the polynomial with variables indexed by a (possibly infinite) set $I$. Given a commutative ring $R$, a set $I$, the polynomial ring $R[x_i \mid i \in I]$ is defined as follows:

For $J \subset I$ finite subset, denote $R_J := R[x_i \mid i \in J]$. Define $\mathcal{P} := \{\text{finite subsets of $I$}\}$, ordered by inclusion. This is a directed set, as for all $M, N \in \mathcal{P}$, we have $M \subseteq I$ and $N \subseteq I$ by definition.

For $J_1 \subseteq J_2$ in $\mathcal{P}$, we have a morphism of $R$-algebras $\varphi_{J_1 J_2}: R_{J_1} \to R_{J_2}$, $x_i \mapsto x_i$ (embeddings). This gives a direct system of $R$-algebras; and we can define $R[x_i \mid i \in I] := \varinjlim (R_J)_{J \in P}$. Notice:

\begin{itemize}
    \item All $\varphi_{J_1 J_2}$ are injective (as they are embeddings), and therefore the map $R_J \to R[x_i \mid i \in I]$ is also injective.
    \item Observe that we have $\bigcup_{J \in \mathcal{P}} R_J$ is a direct limit of the system $(R_J)_{J \in \mathcal{P}}$; and since the direct limit is unique up to isomorphisms for a particular directed system, we have $R[x_i \mid i \in I] = \bigcup_{J \in \mathcal{P}} R_J$.
\end{itemize}

Now prove the theorem:

\begin{proof}[Proof of Theorem \ref{thm: algebraic closure exists}]
    As we have mentioned, the idea of the proof is to construct a direct system of fields, where each $\varphi$ (which is a field extension here) includes the roots of irreducible polynomials in the base field; then taking the direct limit of the system gives the desired algebraic closure. 

    The first step is to construct a field extension $k \ext k_1$ s.t. for all $f \in k[x]$ irreducible, there exists $a \in k_1$ s.t. $f(a) = 0$. Define $A = \{ f \in k[x] \mid f \text{ irreducible} \}$. Define $R = k[y_f \mid f \in A]$ and $\underline{a} = (f(y_f) \mid f \in A)$ where $y_f$ are formal variables indexed by polynomials in $A$; and the parenthesis in $\underline{a}$ implies that this is an ideal generated by such elements. 

    Claim that $\underline{a} \neq R$. First observe that by definition $\underline{a} \subseteq R$, as in particular we have $f \in k[y_f] \subseteq k[y_f \mid f \in A] =: R$. Suppose that $\underline{a} = R$. Then the condition is translated to:
    \begin{equation}\label{eq: condition if a_ = R}
        \tag{$\ast$}        
        \exists r \in \Z_{>0},\ f_1, \dots, f_n \in A,\ g_1, \dots, g_n \in R, \qquad \sum_{i = 1}^r \underbrace{g_i}_{\in R} \underbrace{f_i (y_{f_i})} _{\in \underline{a}} = 1
    \end{equation}
    as the ideal being equal to the whole ring is the same as the ideal containing 1. By Proposition \ref{prop: existence of field extending root of polynomial}, for all $i$ there exists field extension $k \ext k_i$ s.t. there exists $a_i \in k_i$ satisfying $f_i(a_i) = 0$. Since there finitely many (exactly $r$) polynomials, we can do this iteratively and get a field extension $k \ext K$ s.t. there exists field extensions $k_i \ext K$ for all $i$.

    Let $J \subseteq A$ be the finite subset containing all the $f_i$s, and also containing all polynomials $g$ s.t. $y_g$ appears in some $g_i \in k$. Define $\varphi: k[y_g \mid g \in J] \ext K$. $y_{f_i} \mapsto a_i$ for all $i$, and $y_g \mapsto \varepsilon \in K$ which is some value we do not care about. Recall that $a_i$s are defined s.t. $f_i(a_i) = 0$. This is a $k$-algebra morphism. 
    
    Now apply $\varphi$ to Eq. \eqref{eq: condition if a_ = R}. Since $\varphi$ is $k$-linear, we have
    \[
        0 = \sum_{i = 1}^r \varphi(g_i) f_i(a_i) = \varphi(1) = 1
    \] 
    which is a contradiction, as for a field we require $0 \neq 1$. Therefore $\underline{a} \neq R$. Further since $\underline{a}$ is an ideal in $R$, there exists a maximal ideal $M$ in $R$ s.t. $\underline{a} \subseteq R$. Define $k' = R/M$ which is not trivial since $\underline{a} \neq R$. Denote $k_1 := \{ a \in k' \mid \text{$a$ algebraic over $k$} \}$. By Proposition \ref{prop: algebraic closure is a field} this is a subfield of $k'$, and by definition is algebraic over $k$. Now consider $f \in k[x]$ irreducible, and $\overline{y_f} \in k' = R/M$ a formal variable. By definition of algebraic closure, $f(\overline{y_f}) = 0$ in $\bar{k}$; and since $f(y_f) \in \underline{a} \subseteq M$, $\overline{f(y_f)} = f(\overline{y_f}) = 0 \in R/M$, which implies that $\overline{y_f} \in k_1$. Since $k' = R/M$ with $R$ extending elements in the form of $y_f$, every element in $k'$ is in the same as $\overline{y_f}$ for some $f \in A$; and by the previous argument we know $k \ext k'$ is algebraic. That is, every $f \in k[x]$ irreducible has a root in $k_1$.

    Now repeat the process with $k$ replace by $k_1$. Conducting induction gives that for all $f \in k_i[x]$, there exists $a \in k_{i + 1}$, $f(a) = 0$ in $k_{i+1}[x]$. This gives algebraic extensions
    \[
        k \ext k_1 \ext k_2 \ext \cdots
    \]
    Now take $\bar{k} := \varinjlim k_i$. This is a field by Parenthesis \ref{pth: direct limit} satisfying $k \ext k_i \ext \bar{k}$, and by the following remark $\bar{j} = \bigcup_{i \geq 1} k_i$. Check:
    \begin{itemize}
        \item \emph{$\bar{k}$ is algebraic over $k$.} This results from the fact that each of the intermediate extension is algebraic; and the result follows from Proposition \ref{prop: transitivity of algebraic extensions}.
        \item \emph{$\bar{k}$ is algebraically closed.} For all $f \in \bar{k}[x]$, there exists $i$ s.t. $f \in k_{i}[x]$. Then $f$ has a root in $k_{i + 1}$ which can be extended to $K$ by universal property of direct limit.  
    \end{itemize}
\end{proof}

\begin{remark}
    If the field $k$ is infinite, the algebraic closure $\bar{k}$ is of the same cardinality as $k$.
\end{remark}

To address the uniqueness of $\bar{k}$, we prove the following theorem that is more general:

\begin{theorem}\label{thm: algebraic extension can be extended to algebraically closure}
    Given two field extensions $k \ext K$ and $k \ext L$ s.t. $K/k$ is algebraic, and $L$ is algebraically closed. Then there is a morphism of $k$-extensions $K \ext L$, where a morphism of $k$-extensions is a morphism of $k$-algebras which is the identity map when restrict to $k$.
\end{theorem}

To prove the theorem we need Zorn's Lemma:
\begin{lemma}[Zorn]
    Given a nonempty ordered set $(A, \leq)$ s.t. every \underline{chain} in $A$ has an \underline{upper bound} in $A$. Then $a$ has a \underline{maximal element}. The terminologies are:
    \begin{itemize}
        \item A \underline{chain} is a totally ordered subset.
        \item An \underline{upper bound} for $B \subseteq A$ is an element $a \in A$ s.t. for all $b \in B, b \leq a$. 
        \item A \underline{maximal element} in $A$ is some $a \in A$ s.t. for all $a'$ s.t. $a \leq a'$, we have $a' = a$.
    \end{itemize}
\end{lemma}

\begin{proof}[Proof of Theorem \ref{thm: algebraic extension can be extended to algebraically closure}]
    Consider the set $A = \{ (k', \varphi) \}$ where $k'$ is a subfield of $K$ for which there exists extensions $k \ext k' \ext K$; and for $i: k \to k'$, $j: k \to L$ we have $\varphi \circ i = j$. Define the partial order on $A$ be such that $(K', \varphi) \leq (K'', \psi)$ if and only if $K' \subseteq K$, and $\restr{\psi}{k'} = \varphi$.

    Now apply Zorn. Check the followings:
    \begin{itemize}
        \item $A$ is non empty. In particular, $(k, j) \in A$.
        \item Suppose that $B = \{ (K_i, \varphi_i) \mid i \in I \}$ is a chain (totally ordered subset) in $A$, Then $B$ has a maximal element (which also serves as an upper bound) $(K, \varphi)$ given by $K = \bigcup_{i \in I} K_i$ and $\varphi$ be such that $\restr{\varphi}{K_i} = \varphi_i$. Such $\varphi$ exists as $B$ is totally ordered, i.e. for any subset of $B$ the maximal element gives the corresponding $\varphi$.
    \end{itemize}
    Then Zorn's Lemma gives that there exists a maximal element $(K', \varphi) \in A$. Then either:
    \begin{itemize}
        \item $K' = K$. This gives the desired result. 
        \item $K' \neq K$. We seek to find a contradiction. Since $K' \subseteq K$, there exists $a \in K \smallsetminus K'$. Since $K/k$ is algebraic by hypothesis, by Remark \ref{rmk: subextension of algebraic extension is algebraic}, $K/K'$ is also algebraic. Let $f \in K'[x]$ be the minimal polynomial of $a$ over $K'$. Try to extend on $a$: 
        \[
            K' \ext K'[a] \simeq K'(a) \simeq K'[x]/(f)
        \]
        where $K'[a] = K'(a)$ since $a$ is algebraic over $k'$. Since $L$ is algebraically closed, $f$ (considering it in $L[x]$) has a root $b \in L$. Then there exists a unique $K'$-algebra morphism $\varphi': K'(a) \to L$, $a \mapsto b$, which is indeed a $K'$-algebra morphism as any $g \in K[x]$ s.t. $g(a) = 0$ satisfies $f \mid g$, which has $b$ also as a root. But then $(K'(a), \varphi')$ satisfies the condition, which contradicts with the maximality of $(K', \varphi)$.
    \end{itemize}
    Therefore $K' = K$ and we have the desired result.
\end{proof}

\begin{corollary}\label{cor: algebraic closure is unique}
    The algebraic closure of a given field $k$ is unique up to isomorphism.

    Suppose that we have field extensions $k \ext K$ and $k \ext L$, with both $K$ and $L$ algebraically closed. Then applying the above theorem twice gives field extensions $K \ext L$ and $L \ext K$, which are both injective by Proposition \ref{prop: ring homs from a field are injective}. This gives the desired isomorphism.
\end{corollary}

\begin{notation}
    Since the algebraic closures of a specific field $k$ are isomorphic, we can simply refer to it as \emph{the} algebraic closure, and denote it with $\bar{k}$.
\end{notation}

\begin{remark}\label{rmk: misc results on algebraic closure}
    We have the following immediate results:
    \begin{enumerate}[label=\arabic*)]
        \item If $k \ext K$ is algebraic, and $K \ext \overline{K}$ algebraic closure, then $k \ext \overline{K}$ is also an algebraic closure (by considering $f \in k[x]$ as elements of $K[x]$).
        \item If we have $k \ext K$ a field extension, and $K$ is algebraically closed. Then for $k' := \{a \in K \mid \text{$a$ algebraic over $k$}\}$ we have $k'/k$ an algebraic closure. This is clearly algebraic by definition; and it is closed as if $f \in k'[x] \smallsetminus \{0\}$, there exists $a \in K$ s.t. $f(a) = 0$ as $K$ is algebraically closed. Then $a$ is algebraic over $k$ \implies $k' \ext k'(a)$ finite \implies $k' \ext k'(a)$ algebraic \implies $k \ext k' \ext k'(a)$ algebraic. But this gives the fact that $a$ is algebraic over $k$, and since $a \in K$ by definition $a \in k'$.  
    \end{enumerate}
    Notice that we cannot use the same trick in the proof for Theorem \ref{thm: algebraic closure exists} as in that case the extensions $k \ext k_1$ is not necessarily algebraic.
\end{remark}

\textstart
Now we can use the above results to classify finite fields:

Let $K$ be a finite field. Since $\Z \to K$ cannot be injective, $\fchar(K) = p$ for some prime $p$. This gives a field extension $\F_p \ext K$; and since $e = [K : \F_p]$ is finite (since $K$ is finite), $\abs{K} = p^e$ for some $e$. Denote $\overline{K}$ to be the algebraic closure of $K$. By Remark \ref{rmk: misc results on algebraic closure} this is also the algebraic closure of $\F_p$.

\begin{claim}\label{clm: fields with p^e elements}
    $K = \{u \in \overline{K} \mid u^{p^e} = u\}$. This gives the existence of finite fields with order $p^e$.
\end{claim}

\begin{proof}
    Notice that $(K^{\times}, \cdot)$ is a finite group with $(p^e - 1)$ elements. Therefore, for all $u \in L$, $u^{p^e - 1} = 1$, which implies that $u^{p^e} = u$ i.e. $K \subseteq $ RHS. Now consider the polynomial $x^{p^e} - x$ in $\overline{K}[x]$, which has $p^e$ roots since $\overline{K}$ is algebraically closed. $\abs{K} = p^e$ gives the desired equality.
\end{proof}

\begin{proposition}\label{prop: F_pe are isomorphic}
    If $K_1$ and $K_2$ are both fields with $p^e$ elements, then $K_1 \simeq K_2$. Without ambiguity we denote such fields as $\F_{p^e}$.
\end{proposition}

\begin{proof}
    Consider the extensions:

    \begin{minipage}{\linewidth}
        \centering
        \begin{tikzcd}
            \F_p \arrow[rr, hookrightarrow] \arrow[dd, hookrightarrow] & & K_1\arrow[dd, dashed, hookrightarrow, "\varphi"] \\
            & & \\
            K_2 \arrow[rr, hookrightarrow] & & \overline{K_2}
        \end{tikzcd}
    \end{minipage}

    This existence of $\varphi$ is guaranteed by Theorem \ref{thm: algebraic extension can be extended to algebraically closure}. Since $\varphi$ is a ring homomorphism between fields, it is injective; and therefore $\abs{\varphi(K_1)} = \abs{K_1} = p^e$ \implies $\varphi(K_1) = \{i \in \overline{K_2} \mid u^{p^e} = u\}$. But this then coincides with $K_2$.
\end{proof}

\begin{example}\label{ex: f_pe extends to f_pf iff e divides f}
    There exists a ring homomorphism $\F_{p^e} \to \F_{p^f}$ if and only if $e \mid f$. 
\end{example}
    
\begin{proof}
    Verify both implications:
    \begin{itemize}
        \item[$\Rightarrow:$] Since we have the field extension $\F_{p^e} \ext \F_{p^f}$, we can view $\F_{p^f}$ as a $\F_{p^e}$-vector space. As further we have two fields being finite, $\abs{\F_{p^f}} = \left( \abs{\F_{p^e}} \right)^{\dim_{\F_{p^e}} \F_{p^f}}$. In particular this implies that $e \mid f$.
        \item[$\Leftarrow:$] By Claim \ref{clm: fields with p^e elements} we know that the elements in the field $\F_{p^f}$ are those in the set $\{ u \in \overline{\F_{p^f}} \mid u^{p^f} = u \}$. Now Consider the roots of the polynomial $f = x^{p^f} - x$ in $\overline{\F_{p^f}}$. Since $\F_{p^f}$ embeds naturally into its algebraic closure, the image is exactly the roots of $f$. Since $e \mid f$, there exists $n$ s.t. $f = ne$. Notice then that $f$ factors as follows:
        \[
            f = x^{p^f} - x = \left( x^{p^e} - x \right) \left( \sum_{i = 1}^{n-1} p^{i(p^e - 1)} \right)
        \]
        which implies that elements satisfying the relation $u^{p^e} = u$ in particular also satisfy $u^{p^f} = u$; and by Claim \ref{clm: fields with p^e elements} these give the elements in field $\F_{p^e}$. This gives the field extension $\F_{p^e} \ext \F_{p^f} \ext \overline{\F_{p^f}}$ (and also by the uniqueness of algebraic closure (Corollary \ref{cor: algebraic closure is unique}) we have $\overline{\F_{p^e}} = \overline{\F_{p^f}}$).
    \end{itemize}
\end{proof}
    
\begin{example}
    Inside the algebraic closure of $\F_p$, $\overline{\F_p}$, for all $e \geq 1$ we have a unique copy of $\F_{p^e}$; and $\overline{\F_p} = \bigcup_{e \geq 1} \F_{p^e}$.
\end{example}

\begin{proof}
    Proceed to verify the inclusion in both directions:
    \begin{enumerate}
        \item[$\subseteq$] For any element $u \in \overline{\F_p}$, consider its minimal polynomial over $\F_p$. Let $f_u \in \F_p[x]$ be the polynomial of $u$. Then $u \in \F_p(u) \simeq \F_p[x]/(f_u)$. By Remark \ref{rmk: degree of minimal polynomial gives degree of extension} we have $\abs{\F_p(u)} = p^{\deg f} \implies u \in \F_{p^{\deg f}} \subseteq \bigcup_{e \geq 1} \F_{p^e}$. 
        \item[$\supseteq$] By Example \ref{ex: f_pe extends to f_pf iff e divides f} we have the embedding of $\F_{p^e}$ to its algebraic closure; and also such closures can be identified by the uniqueness of algebraic closure and viewing $\F_{p^n}$ as elements satisfying relation $u^{p^n} = u$ by Claim \ref{clm: fields with p^e elements}.
    \end{enumerate}
\end{proof}

\section{The Splitting Field of a Polynomial}

\begin{definition}[Splitting Field]
    Let $k$ be a field, and $f \in k[x] \smallsetminus \{0\}$. A \textbf{splitting field} of $f$ is a field extension $k \ext K$ s.t. 
    \begin{enumerate}[label=\arabic*)]
        \item $f$ factors in $K[x]$ as a product of degree-1 polynomials.
        \item If there exists $K'$ s.t. $k \subseteq K' \subseteq K$ which also satisfies 1), then $K' = K$,
    \end{enumerate}
    Equivalently, for a splitting field of $f$, denoted $K$, we can write $f$ in $K[x]$ as $f = c(x - a_1) \cdots (x - a_n)$ with $c, a_1, \dots, a_n \in K$; and $K = k(a_1, \dots, a_n)$. 
\end{definition}

\begin{example}
    $\C$ is the splitting field of $x^2 + 1$. $\Q(\sqrt{2})$ is the splitting field of $x^2 - 2$.
\end{example}

\textstart
From what we had proved previously (Corollary \ref{cor: algebraic closure is unique}), we know that algebraic closure is unique up to isomorphism, so we would expect the splitting field to be unique up to isomorphisms as well. The followings use (which will also be the case for most proofs regarding existence of extensions) the algebraic closure to construct such extensions.

\begin{theorem}\label{thm: splitting field is unique}
    If $f \in k[x] \smallsetminus \{0\}$, then
    \begin{enumerate}
        \item There exists a splitting field $k \ext K$ of $f$.
        \item If we have $k \ext K$, $k \ext K'$ splitting fields of $f$, then $K \simeq K'$ as $k$-extensions.
    \end{enumerate}
\end{theorem}

\begin{remark}
    If $K$ is the splitting field of $f \in k[x]$, then $K/k$ is a finite extension, as in particular $K$ is generated (as a $k$-vector space) by finitely many algebraic elements over $k$, namely the roots of $f$. By Proposition \ref{prop: finitely generated algebraic extensions are finite} the extension is then finite.
\end{remark}

\begin{proof}[Proof of Theorem \ref{thm: splitting field is unique}]
    Prove the two statements respectively:
    \begin{enumerate}[label=\arabic*)]
        \item Consider $k \ext \bar{k}$ the algebraic closure of $k$. Then there exists $c, a_1, \dots, a_n \in \bar{k}$ s.t. $f = c(x - a_1) \cdots (x - a_n)$. Then $K = k(a_1, \dots, a_n)$ is the splitting field of $f$, by definition.
        \item Let $k \ext K$ and $k \ext K'$ be the splitting fields of $f$. Let $K' \ext \overline{K'}$ be the algebraic closure. By the remark above, $K/k$ is algebraic. Since $\overline{K'}$ is algebraically closed, by Theorem \ref{thm: algebraic extension can be extended to algebraically closure} there exists $\varphi: K \to \overline{K'}$ s.t. $\restr{\varphi}{k} = \restr{\Id}{k}$. Since $K$ and $K'$ are splitting fields of $f$ over $k$, we can write $f = c(x - a_1) \cdots (x - a_n)$ in $K[x]$, and $f = c'(x - a_1') \cdots (x - a_n')$ in $K'[x]$. This gives the similar expression of the fields as in the remark: 
        \[
            K = k(a_1, \dots, a_n), \qquad K' = k(a_1', \dots, a_n')
        \]
        Consider the ring homomorphism $K[x] \ext K'[x] \subseteq \overline{K'}[x]$ induced by $\varphi$, we have
        \[
            f = \varphi(f) = c(x - \varphi(a_1)) \cdots (x - \varphi(a_n))
        \]
        since $f \in k[x]$ and $\varphi$ as a morphism between $k$-extensions is the identity map when restricted to $k$. Therefore $\varphi$ permutes $a_i$s, which gives $\varphi(K) \subseteq K'$ and $\varphi(K') \subseteq K$, i.e. $\varphi(K) = K'$, and $\varphi$ is bijection and thus a ring isomorphism.
    \end{enumerate}
\end{proof}

\section{Separable Extensions}

\begin{definition}[Separable Extension]\label{def: separable extension}
    Let $k \ext K$ be an algebraic field extension. An element $a \in K$ is \textbf{separable over $k$} if its minimal polynomial $f \in k[x]$ satisfies any of the following conditions:
    \begin{enumerate}[label=\arabic*)]
        \item $a$ is not a multiple root of $f$.
        \item $f' \neq 0$.
        \item Either $\fchar(k) = 0$, or $\fchar(k) = p > 0$, and $f \notin k[x^p]$.
    \end{enumerate}
    Such a polynomial $f$ is a \textbf{separable polynomial}. A field extension $k \ext K$ is \textbf{separable} if the minimal polynomial of any element in $K$ is separable. 
\end{definition}

\begin{proposition}
    The three conditions in Definition \ref{def: separable extension} for separability over a field are equivalent.
\end{proposition}

\begin{proof}
    Verify the following implications:
    \begin{itemize}
        \item \emph{2) \implies 1).} Prove the contrapositive. Suppose that $a$ is a root of $f$ with multiplicity at least 2. Then we have $(x - a)^2 \mid f \implies (x - a) \mid f'$, i.e. $f'(a) = 0$. Since $f$ is minimal, $f' \mid f$, and yet we require $\deg f' = \deg f$, which implies that $f = 0$.
        \item \emph{1) \implies 2).} Prove the contrapositive by reversing the logic above. $f$ minimal implies $f'(a) = 0$ if and only if $f \mid f'$. But $f'(a) = 0 \implies (x - a)^2 \mid f$, i.e. $a$ is a multiple root of $f$.
        \item \emph{3) \implies 2).} If $\fchar(k) = 0$, then $f' = 0$ if and only if $f$ is constant, which cannot be a minimal polynomial. If $\fchar(k) = p$, then if $f' = 0$ either $f$ is constant (which cannot be the case) or every term of $f'$ vanishes because of characteristic-$p$, i.e. $f \in k[x^p]$.
        \item \emph{2) \implies 3).} Reverse the logic above and verify by the same computation.
    \end{itemize}
\end{proof}

\begin{remark}
    Notice that the condition 2) $f' \neq 0$ does not depend on whether $f$ splits into degree-1 polynomials. Therefore, if $f \in k[x]$ is separable, then $f$ cannot have any multiple roots in any algebraic closure of $k$.
\end{remark}

\begin{definition}[Perfect Field]
    A field $k$ is \textbf{perfect} if every extension $k \ext K$ is separable.
\end{definition}

\begin{proposition}
    A field $k$ is perfect if and only if $\fchar(k) = 0$, or $\fchar(k) = p$ with $k = k^p$, i.e. the map $\varphi: k \to k, x \mapsto x^p$ is an isomorphism.
\end{proposition}

\begin{proof}
    For the case where $k$ is characteristic-0, any minimal polynomial $f$ must be of degree at least 1, giving $f' \neq 0$ which satisfies condition 2).

    Now consider the case where $\fchar{k} = p$. Show the following two implications:
    \begin{itemize}
        \item \emph{If there exists $a \in k \smallsetminus k^p$, then $k$ is not perfect.} Let $k \ext \bar{k}$ be an algebraic closure, and let $b \in \bar{k}$ be a root of $f = x^p - a$. Claim that $f$ is irreducible in $k$ (which implies that $b$ is not separable over $k$ as it is a multiple root). In $k[x]$, $f = x^p - a = x^p - b^p = (x - b)^p$ since $\fchar{k} = p$. Therefore, if $f = gh$ in $k[x]$ with $\deg g > 0$ and $\deg h > 0$, $g$ must taken the form of $g = c(x - b)^i$ for $1 \leq i \leq p - 1$. But then consider the coefficient of $x$, which gives $cib \in k$. Since $c, i \neq 0$ in $k$, $b \in k$, which is a contradiction.
        \item \emph{If $k = k^p$, then $k$ is perfect.} Suppose that we have the algebraic extension $k \ext K$. Choose $u \in K$ arbitrarily, with minimal polynomial $f$. If $u$ is not separable over $k$, then by definition $f \in k[x^p]$, i.e. we can write $f$ as
        \[
            f = \sum_{i = 0}^n a_i x^{p^i} = \sum_{i = 0}^n b_i^p x^{p^i} = \left( \sum_[i = 1]^n b_i x^i \right)^p
        \]
        where the first equality results from $k = k^p$, and the second equality results from $\fchar{k} = p$. But this contradicts with the irreducibility of $f$.
    \end{itemize}
\end{proof}

\begin{example}
    The followings give some examples of separability of extensions:
    \begin{enumerate}
        \item Every field of characteristic 0 is perfect by the proposition above.
        \item Every algebraically closed field $k$ is perfect, as the only algebraic extension from $k$ is $k \ext k$.
        \item Every finite field $\F_{p^e}$ is perfect, as Claim \ref{clm: fields with p^e elements} gives that $u \in \F_{p^e} \implies u^{p^e} = u$, i.e. $\left(u^{p^{e-1}}\right)^p = u$; and using the proposition above gives the desired result.
        \item Fields in the form of $k(x)$ where $\fchar{k} = p > 0$ is not perfect. Suppose that $x \in \left( k(x) \right)^p$, then $x = \left( \frac{f}{g} \right)^p$ for some $f, g \in K[x]$, i.e. $x g^p = f^p$. Counting the degree of the polynomial on both sides, we have $1 \equiv 0 \mod{p}$, which is a contradiction.
    \end{enumerate}
\end{example}

\begin{proposition}\label{prop: transitivity of separable extensions, backwards}
    If we have field extensions $k \ext k_1 \ext k_2$ s.t. the extension $k \ext k_2$ is separable, then both $k \ext k_1$ and $k_1 \ext k_2$ are separable. 
\end{proposition}

\begin{proof}
    $a \in k_1$ can be considered as an element in $k_2$; and since $k \ext k_2$ is separable, $a$ is not a multiple root of its minimal polynomial over $k$, which implies that $k \ext k_1$ is separable. 

    For the second result, consider $b \in k_2$. Let $g$ be its minimal polynomial over $k$, and $h$ be its minimal polynomial over $k_1$. $h$ being minimal implies that $h \mid g$. Since $k \ext k_2$ is separable, $b$ is not a multiple root of $g$, and therefore is not a multiple root of $h$. Since this holds for all $b \in k_2$, the extension $k_1 \ext k_2$ is also separable.
\end{proof}

\textstart
The converse of the proposition above also holds, but we will prove this later.

THe following results characterize the ``nice'' feature of a field extension being separable:

\begin{definition}[Primitive Element]
    Given a field extension $E/F$, $\alpha \in F$ is a \textbf{primitive element} if $E = F(\alpha)$.
\end{definition}
\nogap
\begin{definition}[Simple Extension]
    A field extension $E/F$ is simple if there exists a primitive element for the extension.
\end{definition}

\begin{theorem}\label{thm: primitive element theorem}
    Every finite separable extension is simple. 
\end{theorem}

\begin{proof}
    Since $K/k$ is finite, there exists a basis of $K$ seen as a $k$-vector field. In particular there exists $a_1, \dots, a_n \in K$ s.t. $K = k(a_1, \dots, a_n)$. Perform induction on $n$:
    \begin{itemize}
        \item \emph{$n = 1$.} Clear.
        \item \emph{$n \geq 2$.} It suffices to show the case for $n = 2$, as since the extension is finite, recursively apply the result for $n = 2$ will give the general result. Suppose that $K = k(a, b)$ for $a, b \in K$. 
        
        If $k$ is a finite field, then $K$ is also finite, which implies that $K^{\times}$ is a cyclic group. Let $c \in K^{\times}$ be a generator of $K^{\times}$. Then we have $K = k(c)$.

        Now suppose that $k$ is finite. We seek to prove a stronger result: if $c_{\lambda} = a \lambda + b$ for $\lambda \in k$, then $c_{\lambda}$ is a primitive element for finitely many $\lambda$. Fix $\lambda \in k$. Suppose that $c_{\lambda}$ is not primitive, i.e. we have $k(c_{\lambda}) \subseteq k(a, b)$ a proper extension. Then we must have $a \notin k(c_{\lambda})$ (as otherwise we get $k(c_{\lambda}) \ni c_{\lambda} - a\lambda = b$). Let $f, g \in k(c_{\lambda})[x]$ be the minimal polynomials of $a$ and $b$ over $k(c_{\lambda})$. Then in the algebraic closure $K \ext \overline{K}$, $f$ and $g$ factors as 
        \[
            f = (x - a_1) \cdots (x - a_n), \qquad g = (x - b_1) \cdots (x - b_m)
        \]
        Without loss of generality assume that $a = a_1$ and $b = b_1$. Since $a \notin k(c_{\lambda})$, $n \geq 2$. Now consider the extensions $k \ext k(c_{\lambda}) \ext k(a, b)$. Since $k \ext k(a, b)$ is separable, $k \ext k(c_{\lambda})$ is also separable, which by definition implies that all the $a_i$s are distinct. Now consider $\varphi: k(c_{\lambda})(a_1) \isom k(c_{\lambda})(a_2)$ which sends $a_1 \mapsto a_2$. We then have the commutative diagram:

        \begin{minipage}{\linewidth}
            \centering
            \begin{tikzcd}
                k(c_{\lambda}) \arrow[rr, hookrightarrow] \arrow[rrdd, hookrightarrow] & & k(c_{\lambda})(a_1) \arrow[dd, "\varphi"] \arrow[rr, hookrightarrow] & & \bar{k} \arrow[dd, "\psi"] \\
                & & & & \\
                & & k(c_{\lambda})(a_2) \arrow[rr, hookrightarrow] & & \bar{k}
            \end{tikzcd}
        \end{minipage}
        This induces an isomorphism $\psi: \bar{k} \to \bar{k}$ via specifying $\psi(b) = \psi(b_1) = b_i$ for some $i$. Since $\psi$ should fix $k(c_[\lambda])$, we require
        \[
            a_2 \lambda + b_i = \psi(a \lambda + b) = \psi(c_{\lambda}) = a \lambda + b \implies \lambda = \frac{b_i - b}{a - a_2}
        \]
        Such lambda exists since $a \neq a_2$ by the previous result that all $a_i$s are distinct; and there are only finitely many of them as there are finitely many distinct $b_i$s.
    \end{itemize}
\end{proof}

\section{Normal Extensions}

\section{Galois Extensions}

\section{Algebraic Independence \& Transcendence Degree$^{\ast}$}

\section{The Fundamental Theorem of Galois Theory}

\section{Norm and Trace Maps}

\section{Solvability by Radicals}

\end{document}
\documentclass{article}
\usepackage{../refalg}

\begin{document}
\Makepagesectionhead{MATH 594}{Finite Fields and Galois Theory}{ARessegetes Stery}

\tableofcontents
\newpage

\section{Review of Ring Theory}

\textstart
The Galois Theory originates from the question: Given a polynomial with coefficients in a field (e.g. $\Q$), we want to understand the property of its solutions, and the algebraic structure in which the root lies. This section lays some fundamental notions and results for introducing the whole theory.

Setup the convention. All rings have unit element 1; and all ring homomorphisms map 1 to 1. Inclusion maps, in particular, implies that all subrings of a certain ring must include 1.

\begin{remark}
    An immediate result is that finite domains are fields.

    Recall that a ring $R$ is a domain if and only if zero does not have non-trivial divisors. That is, for all $x, y \in R \smallsetminus \{0\}$, $xy = 0$. For $a \in R$, $a \neq 0$, consider the map 
    \[
        \varphi: R \to R \qquad x \mapsto ax
    \]
    Since $R$ is a domain $\varphi$ maps nonzero elements to nonzero elements, which implies that there exists some $x_a$ s.t. $a x_a = 1$. This gives an inverse of $a$.
\end{remark}

\begin{proposition}
    If $f: K \to R$ is a ring homomorphism, and $K$ is a field. Then $R \neq \{0\}$ implies that $f$ is injective.
\end{proposition}

\begin{proof}
    Recall that the kernel of a particular ring homomorphism is an ideal. Denote $I = \ker (f) \subseteq K$. Suppose that $a \in I$ s.t. $a$ is nonzero. Then $a^{-1} \in I$ which gives $I = (1) = K \implies f(I) = 0$. But as we requre $f(1) = 1_R \implies 1_R = 0_R$, i.e. $R = \{0\}$ which is a contradiction.
\end{proof}

\begin{corollary}
    In particular, ring homomorphisms between fields (field extensions) $K \to L$ are injective. Not all fields have extensions (ring homomorphisms) between them.
\end{corollary}

\textstart
A class of extensions of which we are particularly interested in, is the extension which gives polynomial a root. $f \in K[x]$ may not have a root; and by extending it $K[x] \ext L[x]$ we may consider roots of $f \in L[x]$ which may have a root.

\begin{notation}
    A field extension $k \ext K$ is also denoted as $K/k$. These two notations will be used interchangeably.
\end{notation}

\begin{proposition}
    Let $k$ be a field, and $R = k[X]$. Let $f \in R \smallsetminus \{0\}$. Then the followings are equivalent:
    \begin{enumerate}[label=\arabic*)]
        \item $f$ is irreducible.
        \item $(f)$ is a prime ideal.
        \item $(f)$ is a maximal ideal.
    \end{enumerate}
\end{proposition}

\begin{proof}
    Prove the implications cyclically:
    \begin{itemize}
        \item \emph{3) \implies 2).} It is a general fact that maximal ideals are prime. Prove the contrapositive: suppose that an ideal $(f) \subset R$ is not prime, then there exists $a, b \in R$ s.t. $ab \in (f)$ and neither $a$ and $b$ are in $(f)$. Then $(f) \subset (a) \subset R$ which implies that $(f)$ is not maximal.
        \item \emph{2) \implies 1).} $(f)$ being a prime ideal in $R$ implies that in particular $(f) \neq R$, i.e. $f$ is not invertible. Suppose that there exists $g, h$ not invertible s.t. $f = gh$. Without loss of generality, assume that $g \in (f)$. Then there exists some $u \in R$ s.t. $g = fu$. Multiply on the right by $h$ gives $f = gh = fuh$ which implies that $h$ is invertible, giving a contradiction.
        \item \emph{3) \implies 1).} $f$ being irreducible implies that $f$ is not invertible, i.e. $(f) \neq R$. Suppose that there exists some maximal ideal $J$ s.t. $(f) \subset J \subset R$. Then since maximal ideals are in particular prime, $J = (g)$ for some $g \in R$. But then this implies that $f = gu$ giving that $u$ is invertible, and therefore $(f) = J$, which is a contradiction.
    \end{itemize}
\end{proof}

\textstart
Recall that two elements $f, g$ are \underline{relative prime} if for all $p \in R$ s.t. $p \mid f$, $p \mid g$, $p$ is invertible. Then we have the following result similar to the case for integer divisibility:

\begin{proposition}\label{prop: divisibility for relative prime polynomials}
    Let $k$ be a field, and $R = k[x]$. For $f, g, h \in R$ s.t. $f \mid gh$, if $f$ and $g$ are relative prime, then $f \mid h$.
\end{proposition}

\begin{proof}
    Since $k$ is a field, $k[x]$ is a PID (as every element in the coefficient is invertible, for $a, b$ relative prime $(a, b) = (1) = k[x]$). Consider $I = (f, g) \subseteq R$. Since $I$ is principal, $I = (p)$ for some $p \in k[x]$. Therefore $p \mid f$, $p \mid g$, which implies that $p$ is invertible. Then $I = R$. This gives that there exists $A, B \in R$ s.t. $Af + Bg = 1$. Multiplying $h$ on the right gives $Afh + Bgh = h$. Since $f$ divides LHS, $f \mid h$.
\end{proof}

\begin{proposition}
    Let $k$ be a field, and $R = k[x]$. If $f \in R \smallsetminus \{0\}$, and denoting $d = \deg f$, then there exists field extensions $k \ext L$ and $a_1, \dots, a_d \in L$, $c \in k$ s.t. $f = c(x - a_1)\cdots (x - a_d)$ in $L[x]$.
\end{proposition}

\begin{proof}
    THe key step is to show that if $f$ is irreducible in $R$, then there exists a field extension $k \ext k'$ s.t. $f$ has a root in $k'$.

    Consider $k' = k[x]/(f)$. Since $f$ is irreducible, $(f)$ is a maximal ideal in $R$, and therefore $k'$ is a field. Since elements in $k$ are of degree 0 in $k[x]$. Considering $k \too k[x] \too k' := k[x]/(f)$ gives the injective ring homomorphism (field extension). Let $a = \overline{x} \in k'$. Then $f(a) = \overline{f(x)} = 0$, which implies that $a$ is a root of $f$.

    Proceed the rest of the proof by induction: 
    \begin{itemize}
        \item \emph{$d = 0$.} This is the trivial case.
        \item \emph{$d = 1$.} Then $f = c(x - a)$ for some $c, a \in k$.
        \item \emph{$d \geq 2$.} Apply the above steps iteratively. Then there exists some $a \in k'$ s.t. $(x - a) \mid f$, i.e. in $k'[x]$ we have the decomposition $f = (x - a) g$ for some $g \in k'[x]$ with $\deg g = \deg f - 1$. Applying the inductive hypothesis (the results holds in lower degrees) gives the full decomposition in $L := k'$. 
    \end{itemize}
\end{proof}

\begin{notation}\label{not: extending field by algebra}
    Denote the image of the map $(k[y] \to k, y \mapsto a)$ by $k[a]$. This is the smallest $k$-algebra containing $a$.
\end{notation}

\begin{proposition}
    Let $k$ be a field, and $R = k[x]$. Let $f \in R \smallsetminus \{0\}$ be irreducible. Suppose that we have the field extension $k \ext K$, and $a \in K$ is a root of $f$. Then $k[a] \simeq k[x]/(f)$. In particular, $k[a]$ is a field.
\end{proposition}

\begin{proof}
    Consider the ring homomorphism $\varphi: k[y] \to K$ s.t. $\varphi(y) = a$. Then by the First Isomorphism Theorem, we have $k[a] = \im \varphi \simeq k[x]/\ker \varphi \simeq k[x]/(f)$ since $f(a) = 0$. 
\end{proof}

\textstart
Notice that every field extension $k \ext K$ is a $k$-algebra morphism (ring homomorphisms that are $k$-linear). Since $k$ is a field, this gives $K$ a $k$-vector space structure.

\begin{definition}[Degree]
    The \textbf{degree} of the field extension $k \ext K$, denoted $[K : k]$, is $\dim_k K \in \Z_{\geq 0}$ or infinite. 
\end{definition}
\nogap
\begin{definition}[Finite]
    A field extension is \textbf{finite} if the degree of it is finite. 
\end{definition}

\begin{remark}\label{rmk: degree of minimal polynomial gives degree of extension}
    If $f \in k[x]$ is irreducible, and $K = k[x]/(f)$, then $[K : k] = \deg f$. More generally, if $g \in k[x]$ is a nonzero polynomial, then $\dim_k(k[x]/(g)) = \deg g$.

    This can be seen via applying the division algorithm (since $K[x]$ is an Euclidean Domain. This can be seen via computing the division). Then for all $P \in k[x]$, there exists unique $Q, R \in k[x]$ s.t. $P = gQ + R$, with $\deg R < \deg g$. Then since $\overline{P} = \overline{R}$ in $k[x]/(g)$, $\{ \bar{1}, \bar{x}, \dots, \overline{x^{\deg g - 1}} \}$ gives a basis of $k[x]/(g)$ over $k$.
\end{remark}

\section{Multiplicity of Root}

\textstart
This section provides tools for describing the zeros of a polynomial, and how they in general can look like. The proposition below says that any polynomial can be factored into two parts, with the first part having roots in the field; and the second part requires extension of the field to decompose completely. 

\begin{definition}[Multiplicity]
    Let $f \in k[x]$ be a nonzero polynomial for $k$ a field, and $a \in R$ a root of $f$, Then $a$ has \textbf{multiplicity} $m$ if $(x - a)^m \mid f$, but $(x - a)^{m + 1} \nmid f$.
\end{definition}

\begin{proposition}
    If $f \in R \smallsetminus \{0\}$, and $a_1, \dots, a_r \in k$ are pairwise distinct roots of $f$ s.t. $a_i$ has multiplicity $m_i$. Then we have the decomposition of $f$: 
    \[
        f = \prod_{i = 1}^r (x - a_i)^{m_i} g, \qquad g \in R, \text{$g(a_i) \neq 0$ for all $i$}
    \]
    In particular, $\sum_{i} m_i \leq \deg f$.
\end{proposition}

\begin{proof}
    Apply induction on $r$: 
    \begin{itemize}
        \item \emph{Base case.} Then $m_1$ is the maximal integer satisfying the condition that $(x - a_1)^{m_1} \mid f$. Then define $g$ be such that $f = (x - a_1)^{m_1} g$. 
        \item \emph{Inductive step.} For $r \geq 2$,  denote $f_1$ be the polynomial s.t. $f = (x - a_1)^{m_1} f_1$. Notice that for all $i$ s.t. $2 \leq i \leq r$, we have $(x - a_i)^{m_i} \mid f$. Then since $(x - a_i)$ and $(x - a_1)$ are relative prime (they are both irreducible) by Proposition \ref{prop: divisibility for relative prime polynomials} we have $(x - a_i)^{m_i} \mid f_1$. Then applying inductive hypothesis gives the desired decomposition of $f$. 
    \end{itemize}
\end{proof}

\section{Characteristic of a Field}

\textstart
Recall that in the first section we mentioned that there does not necessarily exist ring homomorphisms between arbitrary fields. This, as we will see in the following, implies some constraints on the structure that a field can have.

Let $S$ be an integral domain. Let $\varphi: \Z \to S$ s.t. $n \mapsto n \cdot 1_S$. This is the unique ring homomorphism between $\Z$ and $S$ due to the constraint the $1$ should be mapped to 1. Since $S$ is a domain, and $\Z$ is a PID, $\ker \varphi = (d)$ for $d$ prime or zero. Then either 
\begin{enumerate}[label=\arabic*)]
    \item $\ker \varphi = \{0\}$; or
    \item $\ker \varphi = p\Z$ for some $p$ prime.
\end{enumerate}

In case 1), if we suppose further that $S = k$ which is a field, then for all $n \in \Z$ $\varphi(n)$ is invertible. By the universal property of the quotient ring, this induces a ring homomorphism (which is also a field extension) $\Frac(\Z) = \Q \ext S$. 

In case 2), we have an injective ring homomorphism $\Z/p\Z \ext S$ for some $p$ prime. Defining $\F_p = \Z/p\Z$, $S$ becomes an $\F_p$-algebra by the $\varphi$ above. 

\begin{definition}[Characteristic]
    For a field $k$, the \textbf{characteristic} of $k$ is
    \[
        \fchar(k) = 
        \begin{cases}
            p, & \text{if $\F_p \ext k$ (case 1)} \\
            0, & \text{if $\Q \ext k$ (case 2)}
        \end{cases}
    \]
\end{definition}

\begin{remark}
    If $S$ is an $\F_p$-algebra (case 2), the map $F: S \to S$, $u \mapsto u^p$ is the \underline{Frobenius homomorphism}. Check that this is indeed a ring homomorphism:
    \begin{itemize}
        \item $F(uv) = F(u) F(v)$. Clear as field is commutative: $(uv)^p = u^p v^p$.
        \item $F(u + v) = F(u) + F(v)$. Compute: 
        \[
            (u + v)^p = u^p + v^p + \underbrace{\sum_{i = 1}^{p - 1} \binom{p}{i} u^{p-i} v^i}_{\text{divisible by $p$}}
        \]
        where the last term vanishes, as $\F_p \ext S$ should map $0$ to $0$; and $\bar{p} = \bar{0} \in \F_p$.
    \end{itemize}
\end{remark}

\section{Algebraic Extensions}

\textstart
The field extensions originating solely from ``including the roots of polynomials'' are the nice ones and deserve a better name. The discussions formalizes the concept of ``algebraic closure'' in elementary discussions of polynomials.

\begin{proposition}\label{prop: degree of composition of field extensions}
    If $k \ext K \ext L$ is a field extension, then $[L:k] = [L:K][K:k]$.
\end{proposition}

\begin{proof}
    First consider the cases where one of the degrees is infinite: 
    \begin{itemize}
        \item If $[K : k]$ is infinite, then $[L : k]$ is infinite as $K \subseteq L$ is a $K$-vector subspace of $L$. 
        \item If $[L : K]$ is infinite, then there exists an infinite set of elements which are linearly independent over $K$, which are also linearly independent over $k$ since $k \subseteq K$.
    \end{itemize}
    Now consider the case where both $[L:K]$ and $[K:k]$ are finite. Denote $m = [L:K]$ and $n = [K:k]$. Denote $\{a_1, \dots, a_m\}$ be a basis of $L$ over $K$, and $\{b_1, \dots, b_n\}$ be a basis of $K$ over $k$. Notice that $\{a_i b_j \mid 1 \leq i \leq m, 1 \leq j \leq n\}$ gives a basis for $L$ over $k$, as for all $u \in L$, there exists $\lambda_i \in K$, and thus $\mu_{ij} \in k$ s.t.
    \[
        u = \sum_{i = 1}^n \lambda_i b_i = \sum_{i, j} \mu_{ij} a_j b_i, \qquad \text{for $\lambda_i = \sum_{j} \mu_{ij} a_j \mu_{ij}$}
    \]
    which is a decomposition. They are further linearly independent, as for $u = 0$, since $b_i$s give a basis, $\lambda_i = 0$ for all $i$, and therefore $\mu_{ij} = 0$ for all $i$ and $j$.
\end{proof}

\begin{notation}\label{not: extending field by field}
    Let $k \ext K$ bea field extension, and $A \subseteq K$ a subset. Then we denote
    \[
        k(A) := \cap_{A \subseteq k'} \left\{ k' \mid k \ext k' \ext K \text{ extension} \right\}
    \]
    which is the smallest field sub-extension of $k$ inside $K$ containing $A$. 
\end{notation}

\begin{remark}
    It is worth mentioning that this is different from $k[A]$ which is the smallest \emph{$k$-subalgebra} containing $A$:
    \begin{itemize}
        \item They are related via $k(A) = \Frac(k[A])$. They are equal in some ``nice'' extensions (see Remark \ref{rmk: algebraic implies k[a] = k(a)} below).
        
        By definition we have $k[A] \subseteq k(A)$, as $k[A]$ is only required to be a $k$-algebra instead of a field extension of $k$ (as field extending $k$ can be seen as $k$-vector spaces, which are in particular $k$-algebras). By the universal property of fraction fields, we have $\Frac(k[A]) \subseteq k(A)$, as ring homomorphisms between fields are injective, and by definition for all $f \in k[A]$, $f$ has an inverse in $k(A)$. Further since $A \subseteq \Frac(k[A])$ (also by definition, we have $k(A) = \Frac{k[A]}$).
        \item Considering multiple elements, we can extend the \hyperref[not: extending field by algebra]{previous notation}, by considering for $A = \{a_1, \dots, a_n\}$, then $k[A] = \im \varphi$ for
        \[
            \varphi: k[x_1, \dots, x_n] \to K, \qquad x_i \mapsto a_i
        \]
    \end{itemize}
\end{remark}

\begin{definition}[Finite Generated]
    A field extension $K/k$ is \textbf{finitely generated} if there exists $a_1, \dots, a_n \in K$ s.t. \newline $k(a_1, \dots, a_n) = K$.
\end{definition}

\begin{remark}
    If a field extension $K/k$ is finite, then it is also finitely generated, as $K/k$ being finite implies that there exists some finite basis of $K$ over $k$; and picking one gives the elements that ``finitely generates'' $K$. However, the converse is not true: consider $k \ext k(x) = \Frac(k[x])$ is finitely generated (by $x$) but is not finite (we have the infinite set $\{ x^i \mid i \in \Z \}$ whose elements are linearly independent over $k$)
\end{remark}

\begin{definition}[Algebraic; Transcendental]
    Let $k \ext K$ be a field extension. An element $a \in K$ is \textbf{algebraic over $k$} if there exists $f \in k[x] \smallsetminus \{0\}$ s.t. $f(a) = 0$ in $K[x]$. Otherwise $a$ is \textbf{transcendental}. An extension $K/k$ is \textbf{algebraic} if for all $a \in K$, it is algebraic over $k$.
\end{definition}

\begin{remark}
    Consider the field extensions $k \ext K \ext L$. Then if $a \in L$ is algebraic over $k$, then $a$ is also algebraic over $K$, as $a$ algebraic over $k$ implies that there exists $f \in k[x]$ s.t. $f(a) = 0$; and by definition we also have $f \in K[x]$.
\end{remark}

\begin{remark}\label{rmk: kernel of evaluation is prime}
    Given a field extension $k \ext K$, and $a \in K$. Then $a$ is algebraic if and only if the $\varphi: k[x] \to k, x \mapsto a$ has a non-trivial kernel. This is the direct translation of having a polynomial $f$ with $a$ as its root. Then $\ker \varphi$ is a prime ideal.
        
    To prove this, it suffices to show that $k[x]/(\ker \varphi)$ is a domain. This is indeed the case, as $k[x]$ is a domain: for all $g, h \in k[x]$, $g(a) \neq 0$ and $h(a) \neq 0$ implies that $gh(a) \neq 0$, i.e. $gh \notin \ker \varphi$ ($gh \neq \bar{0}$ in $k[x]/(\ker \varphi)$). Therefore, there exists some $f$ s.t. $\ker \varphi = (f)$.
\end{remark}

\begin{definition}[Minimal Polynomial]
    $f \in k[x]$ is the \textbf{minimal polynomial} of $a \in K$ if for $\varphi: k[x] \to K$, $x \mapsto a$, $\ker \varphi = (f)$. This is well-defined by the above Remark (Remark \ref{rmk: kernel of evaluation is prime}).
\end{definition}

\begin{remark}\label{rmk: algebraic implies k[a] = k(a)}
    For $a$ being algebraic, $f$ is a maximal ideal (by the fact that $k[x]/(\ker \varphi)$ is a domain). Then $k[a] = k[x]/(\ker \varphi)$ is a field. This gives $k[a] = k(a)$.
\end{remark}

\begin{remark}
    Given field extension $K/k$, if $a \in K$ is transcendental over $k$, then $k[a] \simeq k[x]$, as since there is no polynomial with root $a$ implies that $k[a]$ can be seen as injecting a formal variable to the field $k[x]$. This further implies $k(a) \simeq k(x)$ (as $k(a) \simeq \Frac(k[a])$; and same for $x$).
\end{remark}

\begin{example}
    Suppose that $d \in \Z$ is not a square, and let $a = \sqrt{d}$. Consider the field extension $\Q \to \C$.
    
    Since $a$ is not in $\Q$, $a$ cannot be a root of degree 1 polynomials, i.e. the minimal polynomial of $a$ must be of degree at least 2; and we have $a$ as a root of $x^2 - d = 0$, the minimal polynomial of $a$ over $k$ is $x^2 - d = 0$, which also implies that $a$ is algebraic over $\Q$.

    Therefore, $\Q(\sqrt{d})$ can be seen as a $\Q$-vector space, which has a basis $\{1, \sqrt{Q}\}$, i.e. $\Q(\sqrt{d}) = \{ a + b \sqrt{d} \mid a, b \in \Q \}$. In particular we have $\dim_{\Q} \Q(\sqrt{d}) = 2$ as $\deg f = 2$, resulting from Remark \ref{rmk: degree of minimal polynomial gives degree of extension}.
\end{example}

\begin{proposition}\label{prop: finite extensions are algebraic}
    Every finite extension $K/k$ is algebraic. 
\end{proposition}

\begin{proof}
    Denote $[K : k] = n$ which is finite. Regarding $K$ as a $k$-vector space, for all $a \in K$ the set of elements $\{ 1, a, \dots, a^n \}$ are linearly dependent (as there are $(n+1)$ of them). That is, there exists $c_0, \dots, c_n \in k$ s.t. $c_0 + c_1 a + \cdots + c_n a^n = 0$; and $f = c_0 + c_1 x + \cdots + c_n x^n \in k[x] \smallsetminus \{0\}$. This gives a polynomial $f \in k[x]$ s.t. $f(a) = 0$, i.e. $a$ is algebraic. Since $a \in K$ can be taken arbitrarily, we have $K/k$ being algebraic. 
\end{proof}

\begin{proposition}\label{prop: finitely generated algebraic extensions are finite}
    Let $k \ext K$ be a field extension, and $a_1, \dots, a_n \in K$ are algebraic over $k$. Then $k[a_1, \dots, a_n] = k(a_1, \dots, a_n)$; and $k(a_1, \dots, a_n)$ is a finite extension over $k$. 
\end{proposition}

\begin{proof}
    Remark \ref{rmk: algebraic implies k[a] = k(a)} gives the case for $n = 1$, $k[a] = k(a)$, and Remark \ref{rmk: degree of minimal polynomial gives degree of extension} (using minimal polynomial) gives that the extension is finite. 

    For $n \geq 2$, repeat the argument with the induction hypothesis. First by definition we have $k[a_1, \dots, a_n] = (k[a_1, \dots, a_{n-1}]) [a_n]$; and the same holds for the field extension version (replace brackets with parentheses). Further finite extensions are transitive, by Proposition \ref{prop: degree of composition of field extensions}. Suppose now that $k' = k[a_1, \dots, a_{n-1}] = k(a_1, \dots, a_{n-1})$ is a finite extension over $k$. Then $k[a_1, \dots, a_n] = k'[a_n] = k'(a_n) = k(a_1, \dots, a_n)$ is finite over $k'$; and by hypothesis we know $k'[a_n]$ is finite over $k$ since $k'$ is. 
\end{proof}

\begin{corollary}
    Finitely generated algebraic field extensions are finite.
\end{corollary}

\begin{proposition}
    If $k \ext K \ext L$ are both algebraic field extensions, then so is $k \ext L$.
\end{proposition}

\begin{proof}
    Use the above two propositions. Let $a \in L$. Since $L/K$ is algebraic, there exists $f \in K[x] \smallsetminus \{0\}$ s.t. $f(a) = 0$. Let $f = c_0 + c_1 x + \cdots + c_n x^n$, and $k' = k(c_0, \dots, c_n)$. Since all $c_0, \dots, c_n \in K$ are algebraic over $k$, Proposition \ref{prop: finitely generated algebraic extensions are finite} implies that $k'$ is a finite extension over $k$. Further since $a$ is algebraic over $k'$, the extension $k' \ext k'(a)$ is also finite. Proposition \ref{prop: finite extensions are algebraic} implies that the extension $k \ext k'(a)$ is algebraic. That is, $a$ is algebraic over $k$.

    Since this holds for all $a \in L$, the extension $L/k$ is algebraic.
\end{proof}

\begin{proposition}\label{prop: algebraic closure is a field}
    If $k \ext K$ is a field extension, then $k' = \{a \in K \mid \text{$a$ algebraic over $k$}\}$ is a subfield of $K$ containing $k$,
\end{proposition}

\begin{proof}
    To prove this we need to check:
    \begin{itemize}
        \item $k \subseteq k'$. This is clear from the construction of the field.
        \item $k'$ is closed under additive and multiplicative inverse. 
        
        $a \in k'$ implies $-a \in k'$ as $-a$ is the root of a polynomial via considering the minimal polynomial and inverting the corresponding coefficients. $a \in k' \smallsetminus \{0\}$ implies $a^{-1} \in k' \smallsetminus \{0\}$ since $k[a] = k(a)$ as $k(a)$ is finitely generated, and therefore finite and algebraic. $k(a)$ is a field, implying that $a^{-1} \in k(a)$.
        \item $k'$ is closed under addition and multiplication. That is, for all $a, b \in k'$, $a + b \in k'$ and $ab \in k'$.

        Consider the field $k(a, b)$. Since both $a$ and $b$ are algebraic over $k$, $k(a, b)/k$ is finite by Proposition \ref{prop: finitely generated algebraic extensions are finite}; and by Proposition \ref{prop: finite extensions are algebraic} the extension is algebraic. By definition $k(a, b) \subseteq k'$; and is the smallest field containing both $a$ and $b$; and therefore both $a + b$ and $ab$ are in $k'$.
    \end{itemize}
\end{proof}

\begin{definition}[Algebraic Closure]
    For a field $k$ with field extension $k \ext K$, the field $k' = \{a \in K \mid \text{$a$ algebraic over $k$}\}$ is the \textbf{algebraic closure} of $k$ in $K$. By the above Proposition \ref{prop: algebraic closure is a field}, this is indeed a field.
\end{definition}
\nogap
\begin{definition}[Algebraically Closed]
    A field $k$ is \textbf{algebraically closed} if every nonzero polynomial over $k$ has a \mbox{root in $k$.}
\end{definition}
% using \mbox to suppress newline (otherwise there will be a lagging $k$ in the new line)

\begin{remark}
    By induction, if $f$ is a nonzero polynomial in $k[x]$ with degree $n$< then there exists $a_1, \dots, a_n \in k$, $c \in k^{\ast}$ s.t. $f = c(x - a_1) \cdots (x - a_k)$. That is, every irreducible polynomial has degree 1.
\end{remark}

\begin{notation}
    For a field $k$, the set of invertible (nonzero) elements in it are often denoted as $k^{\ast}$ or $k^{\times}$.
\end{notation}

\begin{proposition}
    A field $k$ is algebraically closed if and only if for all field extensions $k \ext K$, for all $a \in K \smallsetminus k$, $a$ is transcendental over $k$.
\end{proposition}

\begin{proof}
    Prove implication in two directions:
    \begin{itemize}
        \item[$\Rightarrow$:] Suppose that $k$ is algebraically closed, and $a \in K \smallsetminus k$ algebraic over $k$. Then by definition there exists an irreducible polynomial $f \in k[x]$ s.t. $f(a) = 0$. Since $k$ is algebraically closed, the irreducible polynomials are of degree 1, i.e. $\deg f = 1$. Then $a \in k$, which is a contradiction.
        \item[$\Leftarrow$:] Suppose that for all $a \in K \smallsetminus k$, $a$ is transcendental. Proceed to prove that $k$ is algebraically closed by showing that every irreducible polynomial has a root. 
        
        Consider $f \in k[x] \smallsetminus \{0\}$. Consider $K = k[x]/(f)$. Then since $f(\bar{x}) = 0$, $\bar{x}$ is algebraic over $k$. But since all elements in $K \smallsetminus k$ are transcendental, $\bar{x} \in k$. Then in $K$ we have $\bar{x} - a = 0$, for some $a \in k$. Then $f(x) = x - a$ in $K[x]$ which has a root $a$; and as this holds for all $f$, $k$ is algebraically closed.
    \end{itemize}
\end{proof}

\textstart
In summary, a field $k$ being algebraically closed is equivalent to the following conditions:
\begin{enumerate}[label=\arabic*)]
    \item For all field extensions $k \ext K$, either $k \simeq K$, or the extension is not algebraic. 
    \item For all $f \in k[x] \smallsetminus \{0\}$, $f$ factors as a product of polynomials of degree 1.
    \item Every irreducible $f \in k[x] \smallsetminus \{0\}$ has a root in $k$. 
\end{enumerate}

\section{The Splitting Field of a Polynomial}

\section{Separable Extensions}

\section{Normal Extensions}

\section{Galois Extensions}

\section{Algebraic Independence \& Transcendence Degree$^{\ast}$}

\section{The Fundamental Theorem of Galois Theory}

\section{Norm and Trace Maps}

\section{Solvability by Radicals}

\end{document}
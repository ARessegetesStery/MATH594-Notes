\documentclass{article}
\usepackage{../refalg}

\begin{document}
\Makepagesectionhead{MATH 594}{Representation of Finite Groups}{ARessegetes Stery}

\tableofcontents
\newpage

\section{Complex Representation}

\textstart
The motivation of introducing the representation of $G$ is to have a linearized version of group action on sets. Recall that we have the correspondence between action of $G$ on a set $X$ and group homomorphism $G \to S_x$ where $S_x$ is the group of bijective maps on $S$, with the operation defined as composition. Explicitly, this is given by
\[
    \varphi: G \times X \to X \quad \rightsquigarrow \quad G \to S_x,\ g \mapsto \varphi(g, -) : (X \to X) 
\]
We now give the formal definition on vector spaces:

\begin{definition}[Representation]
    A \textbf{(complex) representation} of a group $G$ is a vector space $V$ over $\C$, together with a group homomorphism 
    \[
        \rho: G \to \GL(V) := \{ \varphi: V \to V \mid \varphi \text{ $\C$-linear isomorphism} \}
    \]
    Equivalently, a representation of $G$ is a vector space over $\C$ with an action of $G$ on $V$ $\rho: G \times V \to V$ s.t. for all $g \in G$, the induced map $\varphi(g, -)$ os $\C$-linear.
\end{definition}

\begin{notation}
    The map $\rho(g, -): V \to V$ is often abbreviated as $\rho_g$. The representation is denoted by $V$ or $\rho$, with $V$ emphasizing the vector space structure.
\end{notation}

\begin{definition}[Degree of Repr.]
    The \textbf{degree} of the representation is $\dim_{\C}V$, with the same notation as above. 
\end{definition}

\textstart
For most of the time, we will only consider the representation of finite groups on finite-dimensional vector spaces.

\begin{remark}
    In general, one can consider representations over other fields than $\C$. The reasons why $\C$ is chosen are the followings:
    \begin{enumerate}[label=\arabic*)]
        \item If $G$ is finite, then $\abs{G} \in \C$ is always invertible.
        \item $\C$ is algebraically closed. The implications include, for example, every linear map has an eigenvalue.
    \end{enumerate}
    These specialties will often appear in subsequent proofs. 
\end{remark}

\begin{definition}[Morphism of Repr.]
    Given two representations of $G$, $V$ and $W$, a \textbf{morphism of representations} (or simply \textbf{$G$-morphism}) is a linear map $f: V \to W$ s.t. $f(gv) = g(fv)$ for all $g \in G$, $v \in V$. This is an \textbf{isomorphism} if $f$ is further bijective.  
\end{definition}

\begin{remark}
    Following from the definitions we have the immediate results:
    \begin{enumerate}[label=\arabic*)]
        \item If $V_1 \tooh{f} V_2 \tooh{g} V_3$ are morphisms of representation, then so is $g \circ f$ since $g(f(hv)) = g(hf(v)) = h(g(f(v)))$ for all $h \in G$, $v \in V$. This gives the morphisms of objects, i.e. representations of $G$ give a category.
        \item If $f: V \to W$ is an isomorphism of representations, then so is $f^{-1}$ (simply by writing the equation for definition in the inverse order).
        \item If $V$ and $W$ are representations of $G$, then $\{ f : V \to W \mid f \text{ $G$-morphism} \} \subseteq \Hom_{\C}(V, W)$ gives a $\C$-vector subspace. This is clear as by the fact that $f$ is linear, $V$ as a representation is closed under addition and scalar multiplication.
    \end{enumerate}
\end{remark}

\begin{remark}\label{rmk: repr isom iff conjugate}
    Given a finite-dimensional representation $\rho: G \to \GL(V)$, choosing a basis $\{e_1, \dots, e_n\}$ of $V$ gives us an isomorphism $V \simeq \C^n$, i.e. we have the description of representations in matrices
    \[
        \rho: G \to \GL(V) \simeq \GL_n(\C), \qquad g \mapsto \rho_g = (a_{ij}(g))
    \]
    Let $A$ be the matrix representation of a morphism of representations $f$. Since it is required that a morphism of representations should be compatible with application of $g \in G$, we have $A \circ \rho_g = \rho_g' \circ A$. This implies that $\rho_g' = A \circ \rho_g \circ A^{-1}$, i.e. two representations are isomorphic if and only if they are conjugate in matrix presentation; and the matrix that describes the conjugation is the same for all elements $g \in G$.
\end{remark}

\begin{definition}[Sub-representation]
    Given a representation $V$ of $G$, a \textbf{sub-representation} of $V$ is a vector space $W \subseteq V$ s.t. $gv \in W$ for all $v \in W, g \in G$. 
\end{definition}

\begin{remark}
    In particular, for $W$ a sub-representation of $V$, it is itself a representation with the map $\rho'$ being $\restr{\rho(-)}{W}$. The inclusion $W \hookrightarrow V$, $x \mapsto X$ is a morphism of representation.
\end{remark}

\section{Interpretation via the Group Algebra}

\textstart
Similar to the case of group action where we interpreted the structure of group action by the group homomorphism $G \to S_x$, we would like to have some equivalence to structures that are more explicit, and easier to analyze. This introduces the following definitions:

\begin{definition}[Group Algebra]
    Let $G$ be a group. Then the \textbf{group algebra over $\C$}, denoted $\C[G]$ is a vector space with a basis $\{ \alpha(g) \mid g \in G \}$ in bijection with elements in $G$ (formally). Endow it with a multiplication $\alpha(g) \cdot \alpha(h) = \alpha(gh)$ compatible with the group structure gives the desired ring structure.
\end{definition}

\begin{remark}
    Verifying the ring axioms, we have the fact that the identity in $\C[G]$ to be $\alpha(e)$. This is in fact a $\C$-algebra, with the associated morphism given by $\C \to \C[G]$. Since the image of it are scalars, it is clearly in the center of the group. 
\end{remark}

\textstart
Notice that $G$ is not necessarily a finite group. Therefore the vector space can be infinite-dimensional, which we have imposed the requirement that every element should be a finite sum of linear combination of basis. In the following deduction, denote $\sum'$ to be the finite sum. 

\begin{proposition}
    The group algebra is well-defined.
\end{proposition}

\begin{proof}
    This is clear for the cases where $G$ is finite. Consider the case where $G$ is infinite. Then by definition of the group algebra, for all $u, v \in \C[G]$, we have their decomposition into elements in the basis:
    \[
        u = \sum_{g \in G}' a_g \alpha(g), \qquad v \in \sum_{g \in G}' b_g \alpha(g)
    \]
    Multiplying these two terms together gives
    \[
        u \cdot v = \sum_{g \in G} \left( \sum_{g_1 g_2 = g}  (a_{g_1} b_{g_2}) \right) \alpha(g)
    \]
    Furthermore there are only finitely many such $a_g$s and $b_g$s being nonzero, implying that there are only finitely many nonzero such products. 
\end{proof}

\begin{notation}
    If $G$ is abelian, and the correspondence of elements in $G$ and in $\C[G]$ is written additively. Instead of $\alpha(g)$ one usually writes $\chi^g$ (with the convention that $\chi^g \cdot \chi^h = \chi^{g + h}$).
\end{notation}

\begin{remark}
    $\C[G]$ is a commutative ring if and only if $G$ is an abelian group. ``Only if'' is clear as if $\C[G]$ is commutative implies for all $g, h \in G$, they commute. ``If'' results from the fact that for every element in $x \in \C[G]$ there exists a scalar $\lambda$ s.t. $\lambda x = \alpha(g)$ for some $g \in G$ as $\C$ is a field.
\end{remark}

\begin{example}
    If $G = (\Z, +)$, identifying $x \leftrightarrow \chi^x$ for $x \in \Z$, we have $\C[G] \simeq \bigoplus_{m ]in \Z} \C\chi^m \simeq S^{-1}\C[x]$ for $S = \inner{x} = \{1, x, x^2, \dots\}$. These are the \underline{Laurent Polynomials}. 
    
    If $G = (\Z/n\Z, +)$, we have the identification $x^n = 1$, giving $\C[G] \simeq \C[x]/(x^n - 1)$.
\end{example}

\begin{proposition}\label{prop: repr of G identifiable with C[G]-modules}
    We have the identification between representations of $G$ and $\C[G]$-modules. Morphisms and sub-objects (sub-representations and submodules) are also in correspondence.
\end{proposition}

\begin{proof}
    It suffices to verify 1), as identifications in 2) and 3) are induced by 1). 
    
    Suppose that $V$ is a representation of $G$, Then $V$ has a structure of $\C[G]$-module, whose addition is the same as in the vector space, and scalar multiplication is given by
    \[
        \left( \sum_{g \in G} (a_g \cdot \alpha(g)) \right) \cdot v = \sum_{g \in G} (a_g \cdot g(v))
    \]
    where the sums are finite. Conversely, if $M$ is a $\C[G]$-module, then it has a vector space structure via considering the action $\C \hookrightarrow \C[G]$ which acts on $M$; and the linear map is given by $(g, -)$, where $(g, x) \mapsto \alpha(g) \cdot x$ as specified by the $\C[G]$ module. 
\end{proof}

\begin{remark}
    In general, for a representation over a field $\mathbb{F}$ of $G$, it can be identified with $\mathbb{F}[G]$.
\end{remark}

\section{Examples of Representations}

\textstart
The following gives some common examples of representations:
\begin{enumerate}[label=\arabic*)]
    \item Suppose that $G$ acts on a set $X$. Let $V$ be the free $\C$-vector space associated to $X$, with basis $\{ \alpha(u) \mid u \in X \}$ in bijection with $X$. Define $G \tooh{\rho} \GL(V), g \mapsto \rho_g$, with $\rho_g(\alpha(u)) = \alpha(gu)$. This is the \underline{permutation representation} associated with $X$ where action of elements in the group corresponds to a permutation of the set. This is essentially just the group action, as the representation is completely fixed via specifying its behavior on elements in $X$ (i.e. with coefficient 1). 
    \item Example 1) applied to the action of $G$ on itself, $G \times G \to G$, $(g, h) \mapsto (gh)$ induces a representation $\C[G]$. This is the \underline{regular representation} of $G$. Viewed under the context of Proposition \ref{prop: repr of G identifiable with C[G]-modules}, this is the standard left $\C[G]$-module structure of itself (rings are left-modules over itself). 
    \item Direct sum of representations. If $\rho_V: G \to \GL(V)$ and $\rho_W: G \to \GL(W)$ are representations of $G$, then we can get a representation $\rho: G \to \GL(V \oplus W)$, given by 
    \[
        \rho_g = (\rho_g^V, \rho_g^W): G \times (V \oplus W) \to (V \oplus W), \quad (g, (v, w)) \mapsto (gv, gw)
    \]
    Under the context of Proposition \ref{prop: repr of G identifiable with C[G]-modules}, this corresponds to the direct sum of modules. 
    \item Tensor product of representations. Suppose that we have $\rho: G \to \GL(V)$ and $\rho': G \to \GL(V')$ two representations of $G$. Then we can have
    \[
        \widetilde{\rho} = \rho \tensor \rho': G \to \GL(V \tensor_{\C} V'), \quad g \mapsto (\rho_g \tensor \rho_g')
    \]
    This is indeed a group homomorphism, as tensor product of maps behave functorially. That is, it commutes with composition of maps by the universal property of tensor product:
    \[
        (f \tensor g) \circ (f' \tensor g') = (f \circ f') \tensor (g \circ g')
    \]
\end{enumerate}

\section{Irreducible Representations}

\textstart
Similar to the introduction of simple groups in group theory, we would like to have some simple objects in terms of representation, s.t. for any representation it can be decomposed into the ``combination'' of these simple objects, and understanding them provides understanding of the whole object. 

Consider the simplest case of representation of $G$, where it is 1-dimensional, and is given by $G \to \C^{\ast} \simeq (\C \to \C)$ as a group homomorphism. The composition of $\C^{\ast}$ is the multiplication, as here $\C$ is considered as the 1-by-1 complex matrix (1-dimensional linear map). Since $\C^{\ast}$ is commutative, this is the same as the representation $\bar{\rho}: G^{\ab} \to \C^{\ast}$. By Remark \ref{rmk: repr isom iff conjugate}, two representations are isomorphic if and only if they are conjugate; and since $\C^{\ast}$ is commutative, this implies that two representations are isomorphic if and only if they are identical.

\begin{corollary}\label{cor: kernel and image of morphism of repr. is repr.}
    If $F: V \to W$ is a morphism of $G$-representations, then $\ker f \subseteq V$ and $\im f \subseteq W$ are sub-representations. This can be seen via using Proposition \ref{prop: repr of G identifiable with C[G]-modules} to identify representations with $\C[G]$-modules, and see that the kernel and image of a morphism of $R$-modules are both submodules. 
\end{corollary}

The following gives some tools for properly define the concept of ``simple'' objects in terms of representations, and decompose complex objects to those simple ones. From now on, we will consider only $G$ being finite groups, and all representations are finite-dimensional.

\begin{parenthesis}\label{pth: equiv in v.s. for internal sum and projection}
    Let $V$ be a vector space, and $W \subseteq V$ a linear subspace. Then giving the followings are equivalent:
    \begin{enumerate}[label=\arabic*)]
        \item A vector subspace $W' \subseteq V$ s.t. $V = W \oplus W'$ which is the \underline{internal direct sum}, i.e. every element in $V$ can be uniquely decomposed into the sum of an element in $W$ and an element in $W'$; and the two vector subspaces $W$ and $W'$ are linearly independent, i.e. $W \cap W' = \{0\}$.
        \item A linear map $p: V \to V$ s.t. $p^2 = p$, and $\im p = W$.
    \end{enumerate}
\end{parenthesis}

\begin{proof}
    Consider implication in two directions:
    \begin{itemize}
        \item \emph{1) \implies 2).} Given $V = W \oplus W'$, we know that for all $v \in V$, there exists unique $w \in W$ and $w' \in W'$ s.t. $v = w + w'$. Define $p: V \to W$ s.t. $p(v) = w \in W$ with $w$ the same as in the decomposition above. It is then clear that $p^2 = p$.
        \item \emph{2) \implies 1).} Define $W = p(V)$, and $W' = \ker p$. Check: $W \cap W' = \{0\}$, as $v \in W' \implies p(v) = 0$; and $v \in W \implies p(v) = v$, which gives $W \cap W' = \{0\}$; and the decomposition can be seen by $v \mapsto (p(v), v - p(v))$.
    \end{itemize}
    It is further clear that these two transforms are inverse to each other, which proves the assertion.
\end{proof}

\textstart
The general result above is also true for representations:
\begin{theorem}\label{thm: decomposition of repr.}
    Let $G$ be a finite group. Let $V$ be a finite-dimensional $\C$-representation of $G$, and $W \subseteq V$ a sub-representation. Then there exists another sub-representation $W'$ of $V$ s.t. $V = W \oplus W'$.
\end{theorem}

\begin{proof}
    First show that we have similar identifications as in the scenario for vector spaces: for $V$ a representation of $G$, and $W \subseteq V$ a sub-representation, then
    \[
        \text{Giving $W'$ sub-repr. s.t. $V = W \oplus W'$} \quad \Longleftrightarrow \quad \text{Giving $p: V \to V$ morphism of repr. s.t. $p^2 = p$, and $\im p = W$}
    \]
    \begin{itemize}
        \item[$\Rightarrow$:] For $W'$ being a sub-representation, it is in particular a vector subspace of $V$. Then by Parenthesis \ref{pth: equiv in v.s. for internal sum and projection} we have $p: V \to V$ linear map which satisfies the desired conditions. Further verify that this is a morphism of representation: for all $v \in V, g \in G$ we have
        \[
            g(p(v)) = g(p(w + w')) = g(p(w)) = g(w) = p(g(w))
        \]
        since $g(w) \in W$ as $W$ is a sub-representation of $V$. 
        \item[$\Leftarrow$:] The decomposition is clear from the result when considering $V, W, W'$ as vector spaces; and the fact that $W'$ is a sub-representation results from the result that $W' = \ker p$ and kernel of morphism of representations is still a representation (Corollary \ref{cor: kernel and image of morphism of repr. is repr.}).
    \end{itemize}
    Approach the proof via providing the construction in RHS. For $W \subseteq V$ being a sub-representation, by Parenthesis \ref{pth: equiv in v.s. for internal sum and projection} we have a linear map $p: V \to V$ s.t. $p^2 = p$ and $\im p = W$. Now seek using $p$ to construct a morphism of representations: Define 
    \[
        \tilde{p}: V \to V, \quad v \mapsto \sum_{g \in G} g^{-1} p(gv)
    \]
    Verify that this is a morphism of representations. For all $v \in V$ and $h \in G$, we have
    \begin{align*}
        \tilde{p}(hv)
        & = \sum_{g \in G} g^{-1} p(ghv) = h \left( \sum_{g \in G} h^{-1} g^{-1} p(ghv) \right) \\
        & = h\left( \sum_{g' \in G} (g')^{-1} p(g'v) \right) & (\text{Use substitution $g' = gh$}) \\
        & = h(\tilde{p}(v))
    \end{align*}
    Check that $\tilde{p}$ also satisfies the other two conditions, i.e. $\tilde{p}^2 = \tilde{p}$, and $\im \tilde{p} = W$. Notice $\tilde{p(hv)} = h\left( \sum_{g' \in G} (g')^{-1} p(g'v) \right)$, where $p(g'v) \in W$; and linear combination of elements in $W$ still remains in $W$. Further $W$ is a sub-representation gives the fact that $W$ is invariant under $g$-actions. Let $W' = \ker \tilde{p}$ finishes the proof.
\end{proof}

\begin{remark}
    In general linear maps between vector spaces, or even endomorphisms on a specific vector space, are not morphisms of representations, as $\rho_g$ as a linear map in general does not commute with $p$ (which is required by the definition of morphism of representations).
\end{remark}

\begin{remark}\label{rmk: counterexample for nonzero char. field or infintie group}
    It is also vital that we require that $G$ is finite; and the field is of characteristic zero. Otherwise the ``averaging'' process where we divide the sum of all possible representations by the order of $G$ is not valid; and in general Theorem \ref{thm: decomposition of repr.} is \emph{not} true for representations over positive-characteristic fields, or for infinite groups.
    
    A good example given in the homeworks is the followings: take $A = \begin{pmatrix} 1 & 1 \\ 0 & 1 \end{pmatrix} \in K^2 = V$ for $K$ some field, and the representation $\rho: G \to \GL(V)$. Such projection $p$ (as in the proof) does not exist for the following cases:
    \begin{itemize}
        \item $K = \C$, $G = \Z$, and $\rho(1) = A$.
        \item $K = \Z/2\Z$, $G = \Z/2\Z$, and $\rho(\bar{1}) = A$.
    \end{itemize} 
\end{remark}

\begin{remark}
    By Proposition \ref{prop: repr of G identifiable with C[G]-modules} we have the identification between $G$-representations on $V$ and $\C[G]$-module structure on $V$. This implies that for $G$ finite, $W \subseteq V$ being a $\C[G]$-module implies that there exists another $\C[G]$-submodule $W'$ s.t. $V = W \oplus W'$.
\end{remark}

\begin{definition}[Irreducible]
    A representation $V$ of $G$ is \textbf{irreducible} if
    \begin{itemize}
        \item $V \neq \{0\}$.
        \item For every representation $W \subseteq V$, either $W = \{0\}$ or $W = V$.
    \end{itemize}
\end{definition}

\begin{corollary}
    By applying iteratively Theorem \ref{thm: decomposition of repr.} for $V$ any representation of a finite group we have the irreducible decomposition $V = W_1 \oplus \dots \oplus W_r$ for $W_i$s irreducible decompositions.
\end{corollary}

\begin{remark}
    In general irreducible representations do not have to be degree-1. The counterexamples provided in Remark \ref{rmk: counterexample for nonzero char. field or infintie group} are good counterexamples. 
\end{remark}

\begin{remark}\label{rmk: decomposition of repr. unique up to isom}
    The $W_i \subseteq V$ as sub-representations of $G$ are not necessarily unique. For example let $V$ being the trivial representation with degree greater than 1; then any decomposition of $V$ into 1-dimensional subspaces gives the irreducible decompositions, and they are not unique as they can be any linearly-independent subspaces.

    However they are unique up to isomorphisms, i.e. given any irreducible representation $W$, and $V = \bigoplus_{i \in I} W_i$, $\#\{ i \mid W_i \simeq W \}$ and $\sum_{W_i \simeq W} W_i$ are independent of the decomposition. That is, the ``space'' that can be represented by $W$ (up to isomorphism) is fixed in for a given $V$. 
\end{remark}

\textstart
The following lemma gives important foundation for computing morphisms between irreducible representations:

\begin{lemma}[Schur]\label{lem: Schur}
    Suppose that $V$ and $W$ are irreducible representations of $G$; and $f: V \to W$ morphism of representations. Then
    \begin{itemize}
        \item If $V \nsimeq W$, then $f = 0$.
        \item If $V \simeq W$, then $f = \lambda\cdot\Id$ for some $\lambda \in \C$.
    \end{itemize}
\end{lemma}

\begin{proof}
    Use the result from Corollary \ref{cor: kernel and image of morphism of repr. is repr.}, that kernel and image of morphism of representations are also representations. Consider $\ker f$ and $\im f$. Since $V$ is irreducible, either $\ker f = V$ (where $f = 0$) or $\ker f = \{0\}$. Similarly either $\im f = \{0\}$ or $\im f = W$.
    
    Since a morphism of representations is in particular a linear map, $\dim \ker f + \dim \im f = \dim V$; and in particular we cannot have both $\im f = \{0\}$ and $\ker f = \{0\}$. Therefore if $\ker f = \{0\}$, $\im f = W$, which implies that $f$ is an isomorphism. Since $\C$ is algebraically closed, we know that $f$ (as a linear map) has an eigenvalue $\lambda$, i.e. $f - \lambda \cdot \Id$ is not injective. But by the fact that $W$ is irreducible, $f - \lambda \cdot \Id = 0$, i.e. $f = \lambda \cdot \Id$. 
\end{proof}

Now we seek to prove the first assertion in Remark \ref{rmk: decomposition of repr. unique up to isom}. First we need to introduce the structure of representations on the linear maps $\Hom_{\C}(V, W)$.

We have seen that $\Hom_{\C}(V, W)$ obtains a vector space structure with addition and scalar multiplication given by the corresponding operation on the output of the map in $W$. Now suppose that $V$ and $W$ are both $G$-representations. Then there exists a natural $G$-representation structure in $\Hom_{\C}(V, W)$, given by 
\[
    (g \varphi)(v) := g\left( \varphi(g^{-1}(v)) \right)
\]
It is clear that this is linear. Check that this is a group homomorphism:
\[
    ((g_1 g_2) \varphi)(v) = (g_1 g_2) (\varphi ( g_2^{-1} g_1^{-1}(v) )) = g_1\left( g_2 (\varphi(g_2^{-1}(g_1^{-1}(v)))) \right) = (g_1 (g_2 \varphi))(v)
\]

\begin{remark}
    For $V$ any $G$-representation, define $V^G := \{ v \in V \mid gv = v \ \forall g \in G \}$ which is the largest trivial sub-representation of $G$. Then $\Hom_{\C}(V, W)$ can be identified with $\{ \varphi: V \to W \mid \varphi \text{ morphism of repr.} \}$. This follows directly from the fact that $g\varphi(g^{-1} -) = \varphi(-)$ implies that $g^{-1}\varphi(-) = \varphi(g^{-1}-)$ which is exactly the definition of morphism of representations.
\end{remark}

\begin{corollary}[Result 1 in Remark \ref{rmk: decomposition of repr. unique up to isom}]
    If $V$ is a $G$-representation with irreducible decompositions $V = W_1 \oplus \dots \oplus W_r$. Then $\#\{ i \mid W_i \simeq W \} = \dim_{\C}\left( \Hom_{\C}(V, W)^G \right)$
\end{corollary}

\begin{proof}
    By the structure of representations in $V$, we have the isomorphism of representations:
    \[
        \Hom_{\C}(W, V)^G \simeq \Hom_{\C}(W, W_1) \oplus \dots \oplus \Hom_{\C}(W, W_r)
    \]
    Schur (Lemma \ref{lem: Schur}) gives 
    \[
        \dim_{\C}(W, W_i)^G = 
        \begin{cases}
            0, & W \nsimeq W_i \\
            1, & W \simeq W_i
        \end{cases}
    \]
    Since we are in the context of vector spaces, we have
    \[
        \dim_{\C}(W, V) = \sum_{i = 1}^r \dim \Hom_{\C} (W, W_i)
    \]
    and summing up the dimensions gives the desired result.
\end{proof}

\section{Character Theory}

\section{Counting Irreducible $G$-Representations}

\end{document}
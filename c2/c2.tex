\documentclass{article}
\usepackage{../refalg}

\begin{document}
\Makepagesectionhead{MATH 594}{Representation of Finite Groups}{ARessegetes Stery}

\tableofcontents
\newpage

\section{Complex Representation}

\textstart
The motivation of introducing the representation of $G$ is to have a linearized version of group action on sets. Recall that we have the correspondence between action of $G$ on a set $X$ and group homomorphism $G \to S_x$ where $S_x$ is the group of bijective maps on $S$, with the operation defined as composition. Explicitly, this is given by
\[
    \varphi: G \times X \to X \quad \rightsquigarrow \quad G \to S_x,\ g \mapsto \varphi(g, -) \in (X \to X) 
\]
We now give the formal definition on vector spaces:

\begin{definition}[Representation]
    A \textbf{(complex) representation} of a group $G$ is a vector space $V$ over $\C$, together with a group homomorphism 
    \[
        \rho: G \to \GL(V) := \{ \varphi: V \to V \mid \varphi \text{ $\C$-linear isomorphism} \}
    \]
    Equivalently, a representation of $G$ is a vector space over $\C$ with an action of $G$ on $V$ $\rho: G \times V \to V$ s.t. for all $g \in G$, the induced map $\varphi(g, -)$ os $\C$-linear.
\end{definition}

\begin{notation}
    The map $\rho(g, -): V \to V$ is often abbreviated as $\rho_g$. The representation is denoted by $V$ or $\rho$, with $V$ emphasizing the vector space structure.
\end{notation}

\begin{definition}[Dimension of Repr.]
    The \textbf{dimension} of the representation is $\dim_{\C}V$, with the same notation as above. 
\end{definition}

\textstart
For most of the time, we will only consider the representation of finite groups on finite-dimensional vector spaces.

\begin{remark}
    In general, one can consider representations over other fields than $\C$. The reasons why $\C$ is chosen are the followings:
    \begin{enumerate}[label=\arabic*)]
        \item If $G$ is finite, then $\abs{G} \in \C$ is always invertible.
        \item $\C$ is algebraically closed. The implications include, for example, every linear map has an eigenvalue.
    \end{enumerate}
    These specialties will often appear in subsequent proofs. 
\end{remark}

\begin{definition}[Morphism of Repr.]
    Given two representations of $G$, $V$ and $W$, a \textbf{morphism of representations} (or simply \textbf{$G$-morphism}) is a linear map $f: V \to W$ s.t. $f(gv) = g(fv)$ for all $g \in G$, $v \in V$. This is an \textbf{isomorphism} if $f$ is further bijective.  
\end{definition}

\begin{remark}
    Following from the definitions we have the immediate results:
    \begin{enumerate}[label=\arabic*)]
        \item If $V_1 \tooh{f} V_2 \tooh{g} V_3$ are morphisms of representation, then so is $g \circ f$ since $g(f(hv)) = g(hf(v)) = h(g(f(v)))$ for all $h \in G$, $v \in V$. This gives the morphisms of objects, i.e. representations of $G$ give a category.
        \item If $f: V \to W$ is an isomorphism of representations, then so is $f^{-1}$ (simply by writing the equation for definition in the inverse order).
        \item If $V$ and $W$ are representations of $G$, then $\{ f : V \to W \mid f \text{ $G$-morphism} \} \subseteq \Hom_{\C}(V, W)$ gives a $\C$-vector subspace. This is clear as by the fact that $f$ is linear, $V$ as a representation is closed under addition and scalar multiplication.
    \end{enumerate}
\end{remark}

\begin{remark}
    Given a finite-dimensional representation $\rho: G \to \GL(V)$, choosing a basis $\{e_1, \dots, e_n\}$ of $V$ gives us an isomorphism $V \simeq \C^n$, i.e. we have the description of representations in matrices
    \[
        \rho: G \to \GL(V) \simeq \GL_n(\C), \qquad g \mapsto \rho_g = (a_{ij}(g))
    \]
    This implies that two representations are isomorphic if and only if there exists some matrix $A \in \GL_n(\C)$ s.t. $(a_{ij}(g)) = A (b_{ij}(g))$. In particular, applying the result twice gives that (with identification of representations and its matrix form) $\rho_g = A \rho_g' A^{-1}$, i.e. conjugate representations are isomorphic. Such morphisms of representations ($A$) are \underline{equivariant}.
\end{remark}

\begin{definition}[Sub-representation]
    Given a representation $V$ of $G$, a \textbf{sub-representation} of $V$ is a vector space $W \subseteq V$ s.t. $gv \in W$ for all $v \in W, g \in G$. 
\end{definition}

\begin{remark}
    In particular, for $W$ a sub-representation of $V$, it is itself a representation with the map $\rho'$ being $\restr{\rho(-)}{W}$. The inclusion $W \hookrightarrow V$, $x \mapsto X$ is a morphism of representation.
\end{remark}

\section{Interpretation via the Group Algebra}

\textstart
Similar to the case of group action where we interpreted the structure of group action by the group homomorphism $G \to S_x$, we would like to have some equivalence to structures that are more explicit, and easier to analyze. This introduces the following definitions:

\begin{definition}[Group Algebra]
    Let $G$ be a group. Then the \textbf{group algebra over $\C$}, denoted $\C[G]$ is a vector space with a basis $\{ \alpha(g) \mid g \in G \}$ in bijection with elements in $G$ (formally). Endow it with a multiplication $\alpha(g) \cdot \alpha(g) = \alpha(gh)$ compatible with the group structure gives the desired ring structure.
\end{definition}

\begin{remark}
    Verifying the ring axioms, we have the fact that the identity in $\C[G]$ to be $\alpha(e)$. This is in fact a $\C$-algebra, with the associated morphism given by $\C \to \C[G]$. Since the image of it are scalars, it is clearly in the center of the group. 
\end{remark}

\textstart
Notice that $G$ is not necessarily a finite group. Therefore the vector space can be infinite-dimensional, which we have imposed the requirement that every element should be a finite sum of linear combination of basis. In the following deduction, denote $\sum'$ to be the finite sum. 

\begin{proposition}
    The group algebra is well-defined.
\end{proposition}

\begin{proof}
    This is clear for the cases where $G$ is finite. Consider the case where $G$ is infinite. Then by definition of the group algebra, for all $u, v \in \C[G]$, we have their decomposition into elements in the basis:
    \[
        u = \sum_{g \in G}' a_g \alpha(g), \qquad v \in \sum_{g \in G}' b_g \alpha(g)
    \]
    Multiplying these two terms together gives
    \[
        u \cdot v = \sum_{g \in G} \left( \sum_{g_1 g_2 = g}  (a_{g_1} b_{g_2}) \right) \alpha(g)
    \]
    Furthermore there are only finitely many such $a_g$s and $b_g$s being nonzero, implying that there are only finitely many nonzero such products. 
\end{proof}

\begin{notation}
    If $G$ is abelian, and the correspondence of elements in $G$ and in $\C[G]$ is written additively. Instead of $\alpha(g)$ one usually writes $\chi^g$ (with the convention that $\chi^g \cdot \chi^h = \chi^{g + h}$).
\end{notation}

\begin{remark}
    $\C[G]$ is a commutative ring if and only if $G$ is an abelian group. ``Only if'' is clear as if $\C[G]$ is commutative implies for all $g, h \in G$, they commute. ``If'' results from the fact that for every element in $x \in \C[G]$ there exists a scalar $\lambda$ s.t. $\lambda x = \alpha(g)$ for some $g \in G$ as $\C$ is a field.
\end{remark}

\begin{example}
    If $G = (\Z, +)$, identifying $x \leftrightarrow \chi^x$ for $x \in \Z$, we have $\C[G] \simeq \bigoplus_{m ]in \Z} \C\chi^m \simeq S^{-1}\C[x]$ for $S = \inner{x} = \{1, x, x^2, \dots\}$. These are the \underline{Laurent Polynomials}. 
    
    If $G = (\Z/n\Z, +)$, we have the identification $x^n = 1$, giving $\C[G] \simeq \C[x]/(x^n - 1)$.
\end{example}

\begin{proposition}\label{prop: repr of G identifiable with C[G]-modules}
    We have the identification between representations of $G$ and $\C[G]$-modules. Morphisms and sub-objects (sub-representations and submodules) are also in correspondence.
\end{proposition}

\begin{proof}
    It suffices to verify 1), as identifications in 2) and 3) are induced by 1). 
    
    Suppose that $V$ is a representation of $G$, Then $V$ has a structure of $\C[G]$-module, whose addition is the same as in the vector space, and scalar multiplication is given by
    \[
        \left( \sum_{g \in G}' (a_g \cdot \alpha(g)) \right) \cdot v = \sum_{g \in G}' (a_g \cdot \alpha(gv))
    \]
    Conversely, if $M$ is a $\C[G]$-module, then it has a vector space structure via considering the action $\C \hookrightarrow \C[G]$ which acts on $M$; and the linear map is given by $(g, -)$, where $(g, x) \mapsto \alpha(g) \cdot x$ as specified by the $\C[G]$ module. 
\end{proof}

\begin{remark}
    In general, for a representation over a field $\mathbb{F}$ of $G$, it can be identified with $\mathbb{F}[G]$.
\end{remark}

\section{Examples of Representations}

\textstart
The following gives some common examples of representations:
\begin{enumerate}[label=\arabic*)]
    \item Suppose that $G$ acts on a set $X$. Let $V$ be the free $\C$-vector space associated to $X$, with basis $\{ \alpha(u) \mid u \in X \}$ in bijection with $X$. Define $G \tooh{\rho} \GL(V), g \mapsto \rho_g$, with $\rho_g(\alpha(u)) = \alpha(gu)$. This is the \underline{permutation representation} associated with $X$. This is essentially just the group action, as the representation is completely fixed via specifying its behavior on elements in $X$ (i.e. with coefficient 1). 
    \item Example 1) applied to the action of $G$ on itself, $G \times G \to G$, $(g, h) \mapsto (gh)$ induces a representation $\C[G]$. This is the \underline{regular representation} of $G$. Viewed under the context of Proposition \ref{prop: repr of G identifiable with C[G]-modules}, this is the standard left $\C[G]$-module structure of itself (rings are left-modules over itself). 
    \item Direct sum of representations. If $\rho_V: G \to \GL(V)$ and $\rho_W: G \to \GL(W)$ are representations of $G$, then we can get a representation $\rho: G \to \GL(V \oplus W)$, given by 
    \[
        \rho_g = (\rho_g^V, \rho_g^W): G \times (V \oplus W) \to (V \oplus W), \quad (g, v \oplus w) \mapsto (gv, gw)
    \]
    Under the context of Proposition \ref{prop: repr of G identifiable with C[G]-modules}, this corresponds to the direct sum of modules. 
    \item Tensor product of representations. Suppose that we have $\rho: G \to \GL(V)$ and $\rho': G \to \GL(V')$ two representations of $G$. Then we can have
    \[
        \widetilde{\rho} = \rho \tensor \rho': G \to \GL(V \tensor_{\C} V'), \quad g \mapsto (\rho_g \tensor \rho_g')
    \]
    This is indeed a group homomorphism, as tensor product of maps behave functorially. That is, it commutes with composition of maps by the universal property of tensor product:
    \[
        (f \tensor g) \circ (f' \tensor g') = (f \circ f') \tensor (g \circ g')
    \]
\end{enumerate}

\section{Irreducible Representations}

\section{Character Theory}

\end{document}